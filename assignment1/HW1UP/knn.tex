\documentclass[11pt]{article}

    \usepackage[breakable]{tcolorbox}
    \usepackage{parskip} % Stop auto-indenting (to mimic markdown behaviour)
    
    \usepackage{iftex}
    \ifPDFTeX
    	\usepackage[T1]{fontenc}
    	\usepackage{mathpazo}
    \else
    	\usepackage{fontspec}
    \fi

    % Basic figure setup, for now with no caption control since it's done
    % automatically by Pandoc (which extracts ![](path) syntax from Markdown).
    \usepackage{graphicx}
    % Maintain compatibility with old templates. Remove in nbconvert 6.0
    \let\Oldincludegraphics\includegraphics
    % Ensure that by default, figures have no caption (until we provide a
    % proper Figure object with a Caption API and a way to capture that
    % in the conversion process - todo).
    \usepackage{caption}
    \DeclareCaptionFormat{nocaption}{}
    \captionsetup{format=nocaption,aboveskip=0pt,belowskip=0pt}

    \usepackage{float}
    \floatplacement{figure}{H} % forces figures to be placed at the correct location
    \usepackage{xcolor} % Allow colors to be defined
    \usepackage{enumerate} % Needed for markdown enumerations to work
    \usepackage{geometry} % Used to adjust the document margins
    \usepackage{amsmath} % Equations
    \usepackage{amssymb} % Equations
    \usepackage{textcomp} % defines textquotesingle
    % Hack from http://tex.stackexchange.com/a/47451/13684:
    \AtBeginDocument{%
        \def\PYZsq{\textquotesingle}% Upright quotes in Pygmentized code
    }
    \usepackage{upquote} % Upright quotes for verbatim code
    \usepackage{eurosym} % defines \euro
    \usepackage[mathletters]{ucs} % Extended unicode (utf-8) support
    \usepackage{fancyvrb} % verbatim replacement that allows latex
    \usepackage{grffile} % extends the file name processing of package graphics 
                         % to support a larger range
    \makeatletter % fix for old versions of grffile with XeLaTeX
    \@ifpackagelater{grffile}{2019/11/01}
    {
      % Do nothing on new versions
    }
    {
      \def\Gread@@xetex#1{%
        \IfFileExists{"\Gin@base".bb}%
        {\Gread@eps{\Gin@base.bb}}%
        {\Gread@@xetex@aux#1}%
      }
    }
    \makeatother
    \usepackage[Export]{adjustbox} % Used to constrain images to a maximum size
    \adjustboxset{max size={0.9\linewidth}{0.9\paperheight}}

    % The hyperref package gives us a pdf with properly built
    % internal navigation ('pdf bookmarks' for the table of contents,
    % internal cross-reference links, web links for URLs, etc.)
    \usepackage{hyperref}
    % The default LaTeX title has an obnoxious amount of whitespace. By default,
    % titling removes some of it. It also provides customization options.
    \usepackage{titling}
    \usepackage{longtable} % longtable support required by pandoc >1.10
    \usepackage{booktabs}  % table support for pandoc > 1.12.2
    \usepackage[inline]{enumitem} % IRkernel/repr support (it uses the enumerate* environment)
    \usepackage[normalem]{ulem} % ulem is needed to support strikethroughs (\sout)
                                % normalem makes italics be italics, not underlines
    \usepackage{mathrsfs}
    

    
    % Colors for the hyperref package
    \definecolor{urlcolor}{rgb}{0,.145,.698}
    \definecolor{linkcolor}{rgb}{.71,0.21,0.01}
    \definecolor{citecolor}{rgb}{.12,.54,.11}

    % ANSI colors
    \definecolor{ansi-black}{HTML}{3E424D}
    \definecolor{ansi-black-intense}{HTML}{282C36}
    \definecolor{ansi-red}{HTML}{E75C58}
    \definecolor{ansi-red-intense}{HTML}{B22B31}
    \definecolor{ansi-green}{HTML}{00A250}
    \definecolor{ansi-green-intense}{HTML}{007427}
    \definecolor{ansi-yellow}{HTML}{DDB62B}
    \definecolor{ansi-yellow-intense}{HTML}{B27D12}
    \definecolor{ansi-blue}{HTML}{208FFB}
    \definecolor{ansi-blue-intense}{HTML}{0065CA}
    \definecolor{ansi-magenta}{HTML}{D160C4}
    \definecolor{ansi-magenta-intense}{HTML}{A03196}
    \definecolor{ansi-cyan}{HTML}{60C6C8}
    \definecolor{ansi-cyan-intense}{HTML}{258F8F}
    \definecolor{ansi-white}{HTML}{C5C1B4}
    \definecolor{ansi-white-intense}{HTML}{A1A6B2}
    \definecolor{ansi-default-inverse-fg}{HTML}{FFFFFF}
    \definecolor{ansi-default-inverse-bg}{HTML}{000000}

    % common color for the border for error outputs.
    \definecolor{outerrorbackground}{HTML}{FFDFDF}

    % commands and environments needed by pandoc snippets
    % extracted from the output of `pandoc -s`
    \providecommand{\tightlist}{%
      \setlength{\itemsep}{0pt}\setlength{\parskip}{0pt}}
    \DefineVerbatimEnvironment{Highlighting}{Verbatim}{commandchars=\\\{\}}
    % Add ',fontsize=\small' for more characters per line
    \newenvironment{Shaded}{}{}
    \newcommand{\KeywordTok}[1]{\textcolor[rgb]{0.00,0.44,0.13}{\textbf{{#1}}}}
    \newcommand{\DataTypeTok}[1]{\textcolor[rgb]{0.56,0.13,0.00}{{#1}}}
    \newcommand{\DecValTok}[1]{\textcolor[rgb]{0.25,0.63,0.44}{{#1}}}
    \newcommand{\BaseNTok}[1]{\textcolor[rgb]{0.25,0.63,0.44}{{#1}}}
    \newcommand{\FloatTok}[1]{\textcolor[rgb]{0.25,0.63,0.44}{{#1}}}
    \newcommand{\CharTok}[1]{\textcolor[rgb]{0.25,0.44,0.63}{{#1}}}
    \newcommand{\StringTok}[1]{\textcolor[rgb]{0.25,0.44,0.63}{{#1}}}
    \newcommand{\CommentTok}[1]{\textcolor[rgb]{0.38,0.63,0.69}{\textit{{#1}}}}
    \newcommand{\OtherTok}[1]{\textcolor[rgb]{0.00,0.44,0.13}{{#1}}}
    \newcommand{\AlertTok}[1]{\textcolor[rgb]{1.00,0.00,0.00}{\textbf{{#1}}}}
    \newcommand{\FunctionTok}[1]{\textcolor[rgb]{0.02,0.16,0.49}{{#1}}}
    \newcommand{\RegionMarkerTok}[1]{{#1}}
    \newcommand{\ErrorTok}[1]{\textcolor[rgb]{1.00,0.00,0.00}{\textbf{{#1}}}}
    \newcommand{\NormalTok}[1]{{#1}}
    
    % Additional commands for more recent versions of Pandoc
    \newcommand{\ConstantTok}[1]{\textcolor[rgb]{0.53,0.00,0.00}{{#1}}}
    \newcommand{\SpecialCharTok}[1]{\textcolor[rgb]{0.25,0.44,0.63}{{#1}}}
    \newcommand{\VerbatimStringTok}[1]{\textcolor[rgb]{0.25,0.44,0.63}{{#1}}}
    \newcommand{\SpecialStringTok}[1]{\textcolor[rgb]{0.73,0.40,0.53}{{#1}}}
    \newcommand{\ImportTok}[1]{{#1}}
    \newcommand{\DocumentationTok}[1]{\textcolor[rgb]{0.73,0.13,0.13}{\textit{{#1}}}}
    \newcommand{\AnnotationTok}[1]{\textcolor[rgb]{0.38,0.63,0.69}{\textbf{\textit{{#1}}}}}
    \newcommand{\CommentVarTok}[1]{\textcolor[rgb]{0.38,0.63,0.69}{\textbf{\textit{{#1}}}}}
    \newcommand{\VariableTok}[1]{\textcolor[rgb]{0.10,0.09,0.49}{{#1}}}
    \newcommand{\ControlFlowTok}[1]{\textcolor[rgb]{0.00,0.44,0.13}{\textbf{{#1}}}}
    \newcommand{\OperatorTok}[1]{\textcolor[rgb]{0.40,0.40,0.40}{{#1}}}
    \newcommand{\BuiltInTok}[1]{{#1}}
    \newcommand{\ExtensionTok}[1]{{#1}}
    \newcommand{\PreprocessorTok}[1]{\textcolor[rgb]{0.74,0.48,0.00}{{#1}}}
    \newcommand{\AttributeTok}[1]{\textcolor[rgb]{0.49,0.56,0.16}{{#1}}}
    \newcommand{\InformationTok}[1]{\textcolor[rgb]{0.38,0.63,0.69}{\textbf{\textit{{#1}}}}}
    \newcommand{\WarningTok}[1]{\textcolor[rgb]{0.38,0.63,0.69}{\textbf{\textit{{#1}}}}}
    
    
    % Define a nice break command that doesn't care if a line doesn't already
    % exist.
    \def\br{\hspace*{\fill} \\* }
    % Math Jax compatibility definitions
    \def\gt{>}
    \def\lt{<}
    \let\Oldtex\TeX
    \let\Oldlatex\LaTeX
    \renewcommand{\TeX}{\textrm{\Oldtex}}
    \renewcommand{\LaTeX}{\textrm{\Oldlatex}}
    % Document parameters
    % Document title
    \title{knn}
    
    
    
    
    
% Pygments definitions
\makeatletter
\def\PY@reset{\let\PY@it=\relax \let\PY@bf=\relax%
    \let\PY@ul=\relax \let\PY@tc=\relax%
    \let\PY@bc=\relax \let\PY@ff=\relax}
\def\PY@tok#1{\csname PY@tok@#1\endcsname}
\def\PY@toks#1+{\ifx\relax#1\empty\else%
    \PY@tok{#1}\expandafter\PY@toks\fi}
\def\PY@do#1{\PY@bc{\PY@tc{\PY@ul{%
    \PY@it{\PY@bf{\PY@ff{#1}}}}}}}
\def\PY#1#2{\PY@reset\PY@toks#1+\relax+\PY@do{#2}}

\expandafter\def\csname PY@tok@w\endcsname{\def\PY@tc##1{\textcolor[rgb]{0.73,0.73,0.73}{##1}}}
\expandafter\def\csname PY@tok@c\endcsname{\let\PY@it=\textit\def\PY@tc##1{\textcolor[rgb]{0.25,0.50,0.50}{##1}}}
\expandafter\def\csname PY@tok@cp\endcsname{\def\PY@tc##1{\textcolor[rgb]{0.74,0.48,0.00}{##1}}}
\expandafter\def\csname PY@tok@k\endcsname{\let\PY@bf=\textbf\def\PY@tc##1{\textcolor[rgb]{0.00,0.50,0.00}{##1}}}
\expandafter\def\csname PY@tok@kp\endcsname{\def\PY@tc##1{\textcolor[rgb]{0.00,0.50,0.00}{##1}}}
\expandafter\def\csname PY@tok@kt\endcsname{\def\PY@tc##1{\textcolor[rgb]{0.69,0.00,0.25}{##1}}}
\expandafter\def\csname PY@tok@o\endcsname{\def\PY@tc##1{\textcolor[rgb]{0.40,0.40,0.40}{##1}}}
\expandafter\def\csname PY@tok@ow\endcsname{\let\PY@bf=\textbf\def\PY@tc##1{\textcolor[rgb]{0.67,0.13,1.00}{##1}}}
\expandafter\def\csname PY@tok@nb\endcsname{\def\PY@tc##1{\textcolor[rgb]{0.00,0.50,0.00}{##1}}}
\expandafter\def\csname PY@tok@nf\endcsname{\def\PY@tc##1{\textcolor[rgb]{0.00,0.00,1.00}{##1}}}
\expandafter\def\csname PY@tok@nc\endcsname{\let\PY@bf=\textbf\def\PY@tc##1{\textcolor[rgb]{0.00,0.00,1.00}{##1}}}
\expandafter\def\csname PY@tok@nn\endcsname{\let\PY@bf=\textbf\def\PY@tc##1{\textcolor[rgb]{0.00,0.00,1.00}{##1}}}
\expandafter\def\csname PY@tok@ne\endcsname{\let\PY@bf=\textbf\def\PY@tc##1{\textcolor[rgb]{0.82,0.25,0.23}{##1}}}
\expandafter\def\csname PY@tok@nv\endcsname{\def\PY@tc##1{\textcolor[rgb]{0.10,0.09,0.49}{##1}}}
\expandafter\def\csname PY@tok@no\endcsname{\def\PY@tc##1{\textcolor[rgb]{0.53,0.00,0.00}{##1}}}
\expandafter\def\csname PY@tok@nl\endcsname{\def\PY@tc##1{\textcolor[rgb]{0.63,0.63,0.00}{##1}}}
\expandafter\def\csname PY@tok@ni\endcsname{\let\PY@bf=\textbf\def\PY@tc##1{\textcolor[rgb]{0.60,0.60,0.60}{##1}}}
\expandafter\def\csname PY@tok@na\endcsname{\def\PY@tc##1{\textcolor[rgb]{0.49,0.56,0.16}{##1}}}
\expandafter\def\csname PY@tok@nt\endcsname{\let\PY@bf=\textbf\def\PY@tc##1{\textcolor[rgb]{0.00,0.50,0.00}{##1}}}
\expandafter\def\csname PY@tok@nd\endcsname{\def\PY@tc##1{\textcolor[rgb]{0.67,0.13,1.00}{##1}}}
\expandafter\def\csname PY@tok@s\endcsname{\def\PY@tc##1{\textcolor[rgb]{0.73,0.13,0.13}{##1}}}
\expandafter\def\csname PY@tok@sd\endcsname{\let\PY@it=\textit\def\PY@tc##1{\textcolor[rgb]{0.73,0.13,0.13}{##1}}}
\expandafter\def\csname PY@tok@si\endcsname{\let\PY@bf=\textbf\def\PY@tc##1{\textcolor[rgb]{0.73,0.40,0.53}{##1}}}
\expandafter\def\csname PY@tok@se\endcsname{\let\PY@bf=\textbf\def\PY@tc##1{\textcolor[rgb]{0.73,0.40,0.13}{##1}}}
\expandafter\def\csname PY@tok@sr\endcsname{\def\PY@tc##1{\textcolor[rgb]{0.73,0.40,0.53}{##1}}}
\expandafter\def\csname PY@tok@ss\endcsname{\def\PY@tc##1{\textcolor[rgb]{0.10,0.09,0.49}{##1}}}
\expandafter\def\csname PY@tok@sx\endcsname{\def\PY@tc##1{\textcolor[rgb]{0.00,0.50,0.00}{##1}}}
\expandafter\def\csname PY@tok@m\endcsname{\def\PY@tc##1{\textcolor[rgb]{0.40,0.40,0.40}{##1}}}
\expandafter\def\csname PY@tok@gh\endcsname{\let\PY@bf=\textbf\def\PY@tc##1{\textcolor[rgb]{0.00,0.00,0.50}{##1}}}
\expandafter\def\csname PY@tok@gu\endcsname{\let\PY@bf=\textbf\def\PY@tc##1{\textcolor[rgb]{0.50,0.00,0.50}{##1}}}
\expandafter\def\csname PY@tok@gd\endcsname{\def\PY@tc##1{\textcolor[rgb]{0.63,0.00,0.00}{##1}}}
\expandafter\def\csname PY@tok@gi\endcsname{\def\PY@tc##1{\textcolor[rgb]{0.00,0.63,0.00}{##1}}}
\expandafter\def\csname PY@tok@gr\endcsname{\def\PY@tc##1{\textcolor[rgb]{1.00,0.00,0.00}{##1}}}
\expandafter\def\csname PY@tok@ge\endcsname{\let\PY@it=\textit}
\expandafter\def\csname PY@tok@gs\endcsname{\let\PY@bf=\textbf}
\expandafter\def\csname PY@tok@gp\endcsname{\let\PY@bf=\textbf\def\PY@tc##1{\textcolor[rgb]{0.00,0.00,0.50}{##1}}}
\expandafter\def\csname PY@tok@go\endcsname{\def\PY@tc##1{\textcolor[rgb]{0.53,0.53,0.53}{##1}}}
\expandafter\def\csname PY@tok@gt\endcsname{\def\PY@tc##1{\textcolor[rgb]{0.00,0.27,0.87}{##1}}}
\expandafter\def\csname PY@tok@err\endcsname{\def\PY@bc##1{\setlength{\fboxsep}{0pt}\fcolorbox[rgb]{1.00,0.00,0.00}{1,1,1}{\strut ##1}}}
\expandafter\def\csname PY@tok@kc\endcsname{\let\PY@bf=\textbf\def\PY@tc##1{\textcolor[rgb]{0.00,0.50,0.00}{##1}}}
\expandafter\def\csname PY@tok@kd\endcsname{\let\PY@bf=\textbf\def\PY@tc##1{\textcolor[rgb]{0.00,0.50,0.00}{##1}}}
\expandafter\def\csname PY@tok@kn\endcsname{\let\PY@bf=\textbf\def\PY@tc##1{\textcolor[rgb]{0.00,0.50,0.00}{##1}}}
\expandafter\def\csname PY@tok@kr\endcsname{\let\PY@bf=\textbf\def\PY@tc##1{\textcolor[rgb]{0.00,0.50,0.00}{##1}}}
\expandafter\def\csname PY@tok@bp\endcsname{\def\PY@tc##1{\textcolor[rgb]{0.00,0.50,0.00}{##1}}}
\expandafter\def\csname PY@tok@fm\endcsname{\def\PY@tc##1{\textcolor[rgb]{0.00,0.00,1.00}{##1}}}
\expandafter\def\csname PY@tok@vc\endcsname{\def\PY@tc##1{\textcolor[rgb]{0.10,0.09,0.49}{##1}}}
\expandafter\def\csname PY@tok@vg\endcsname{\def\PY@tc##1{\textcolor[rgb]{0.10,0.09,0.49}{##1}}}
\expandafter\def\csname PY@tok@vi\endcsname{\def\PY@tc##1{\textcolor[rgb]{0.10,0.09,0.49}{##1}}}
\expandafter\def\csname PY@tok@vm\endcsname{\def\PY@tc##1{\textcolor[rgb]{0.10,0.09,0.49}{##1}}}
\expandafter\def\csname PY@tok@sa\endcsname{\def\PY@tc##1{\textcolor[rgb]{0.73,0.13,0.13}{##1}}}
\expandafter\def\csname PY@tok@sb\endcsname{\def\PY@tc##1{\textcolor[rgb]{0.73,0.13,0.13}{##1}}}
\expandafter\def\csname PY@tok@sc\endcsname{\def\PY@tc##1{\textcolor[rgb]{0.73,0.13,0.13}{##1}}}
\expandafter\def\csname PY@tok@dl\endcsname{\def\PY@tc##1{\textcolor[rgb]{0.73,0.13,0.13}{##1}}}
\expandafter\def\csname PY@tok@s2\endcsname{\def\PY@tc##1{\textcolor[rgb]{0.73,0.13,0.13}{##1}}}
\expandafter\def\csname PY@tok@sh\endcsname{\def\PY@tc##1{\textcolor[rgb]{0.73,0.13,0.13}{##1}}}
\expandafter\def\csname PY@tok@s1\endcsname{\def\PY@tc##1{\textcolor[rgb]{0.73,0.13,0.13}{##1}}}
\expandafter\def\csname PY@tok@mb\endcsname{\def\PY@tc##1{\textcolor[rgb]{0.40,0.40,0.40}{##1}}}
\expandafter\def\csname PY@tok@mf\endcsname{\def\PY@tc##1{\textcolor[rgb]{0.40,0.40,0.40}{##1}}}
\expandafter\def\csname PY@tok@mh\endcsname{\def\PY@tc##1{\textcolor[rgb]{0.40,0.40,0.40}{##1}}}
\expandafter\def\csname PY@tok@mi\endcsname{\def\PY@tc##1{\textcolor[rgb]{0.40,0.40,0.40}{##1}}}
\expandafter\def\csname PY@tok@il\endcsname{\def\PY@tc##1{\textcolor[rgb]{0.40,0.40,0.40}{##1}}}
\expandafter\def\csname PY@tok@mo\endcsname{\def\PY@tc##1{\textcolor[rgb]{0.40,0.40,0.40}{##1}}}
\expandafter\def\csname PY@tok@ch\endcsname{\let\PY@it=\textit\def\PY@tc##1{\textcolor[rgb]{0.25,0.50,0.50}{##1}}}
\expandafter\def\csname PY@tok@cm\endcsname{\let\PY@it=\textit\def\PY@tc##1{\textcolor[rgb]{0.25,0.50,0.50}{##1}}}
\expandafter\def\csname PY@tok@cpf\endcsname{\let\PY@it=\textit\def\PY@tc##1{\textcolor[rgb]{0.25,0.50,0.50}{##1}}}
\expandafter\def\csname PY@tok@c1\endcsname{\let\PY@it=\textit\def\PY@tc##1{\textcolor[rgb]{0.25,0.50,0.50}{##1}}}
\expandafter\def\csname PY@tok@cs\endcsname{\let\PY@it=\textit\def\PY@tc##1{\textcolor[rgb]{0.25,0.50,0.50}{##1}}}

\def\PYZbs{\char`\\}
\def\PYZus{\char`\_}
\def\PYZob{\char`\{}
\def\PYZcb{\char`\}}
\def\PYZca{\char`\^}
\def\PYZam{\char`\&}
\def\PYZlt{\char`\<}
\def\PYZgt{\char`\>}
\def\PYZsh{\char`\#}
\def\PYZpc{\char`\%}
\def\PYZdl{\char`\$}
\def\PYZhy{\char`\-}
\def\PYZsq{\char`\'}
\def\PYZdq{\char`\"}
\def\PYZti{\char`\~}
% for compatibility with earlier versions
\def\PYZat{@}
\def\PYZlb{[}
\def\PYZrb{]}
\makeatother


    % For linebreaks inside Verbatim environment from package fancyvrb. 
    \makeatletter
        \newbox\Wrappedcontinuationbox 
        \newbox\Wrappedvisiblespacebox 
        \newcommand*\Wrappedvisiblespace {\textcolor{red}{\textvisiblespace}} 
        \newcommand*\Wrappedcontinuationsymbol {\textcolor{red}{\llap{\tiny$\m@th\hookrightarrow$}}} 
        \newcommand*\Wrappedcontinuationindent {3ex } 
        \newcommand*\Wrappedafterbreak {\kern\Wrappedcontinuationindent\copy\Wrappedcontinuationbox} 
        % Take advantage of the already applied Pygments mark-up to insert 
        % potential linebreaks for TeX processing. 
        %        {, <, #, %, $, ' and ": go to next line. 
        %        _, }, ^, &, >, - and ~: stay at end of broken line. 
        % Use of \textquotesingle for straight quote. 
        \newcommand*\Wrappedbreaksatspecials {% 
            \def\PYGZus{\discretionary{\char`\_}{\Wrappedafterbreak}{\char`\_}}% 
            \def\PYGZob{\discretionary{}{\Wrappedafterbreak\char`\{}{\char`\{}}% 
            \def\PYGZcb{\discretionary{\char`\}}{\Wrappedafterbreak}{\char`\}}}% 
            \def\PYGZca{\discretionary{\char`\^}{\Wrappedafterbreak}{\char`\^}}% 
            \def\PYGZam{\discretionary{\char`\&}{\Wrappedafterbreak}{\char`\&}}% 
            \def\PYGZlt{\discretionary{}{\Wrappedafterbreak\char`\<}{\char`\<}}% 
            \def\PYGZgt{\discretionary{\char`\>}{\Wrappedafterbreak}{\char`\>}}% 
            \def\PYGZsh{\discretionary{}{\Wrappedafterbreak\char`\#}{\char`\#}}% 
            \def\PYGZpc{\discretionary{}{\Wrappedafterbreak\char`\%}{\char`\%}}% 
            \def\PYGZdl{\discretionary{}{\Wrappedafterbreak\char`\$}{\char`\$}}% 
            \def\PYGZhy{\discretionary{\char`\-}{\Wrappedafterbreak}{\char`\-}}% 
            \def\PYGZsq{\discretionary{}{\Wrappedafterbreak\textquotesingle}{\textquotesingle}}% 
            \def\PYGZdq{\discretionary{}{\Wrappedafterbreak\char`\"}{\char`\"}}% 
            \def\PYGZti{\discretionary{\char`\~}{\Wrappedafterbreak}{\char`\~}}% 
        } 
        % Some characters . , ; ? ! / are not pygmentized. 
        % This macro makes them "active" and they will insert potential linebreaks 
        \newcommand*\Wrappedbreaksatpunct {% 
            \lccode`\~`\.\lowercase{\def~}{\discretionary{\hbox{\char`\.}}{\Wrappedafterbreak}{\hbox{\char`\.}}}% 
            \lccode`\~`\,\lowercase{\def~}{\discretionary{\hbox{\char`\,}}{\Wrappedafterbreak}{\hbox{\char`\,}}}% 
            \lccode`\~`\;\lowercase{\def~}{\discretionary{\hbox{\char`\;}}{\Wrappedafterbreak}{\hbox{\char`\;}}}% 
            \lccode`\~`\:\lowercase{\def~}{\discretionary{\hbox{\char`\:}}{\Wrappedafterbreak}{\hbox{\char`\:}}}% 
            \lccode`\~`\?\lowercase{\def~}{\discretionary{\hbox{\char`\?}}{\Wrappedafterbreak}{\hbox{\char`\?}}}% 
            \lccode`\~`\!\lowercase{\def~}{\discretionary{\hbox{\char`\!}}{\Wrappedafterbreak}{\hbox{\char`\!}}}% 
            \lccode`\~`\/\lowercase{\def~}{\discretionary{\hbox{\char`\/}}{\Wrappedafterbreak}{\hbox{\char`\/}}}% 
            \catcode`\.\active
            \catcode`\,\active 
            \catcode`\;\active
            \catcode`\:\active
            \catcode`\?\active
            \catcode`\!\active
            \catcode`\/\active 
            \lccode`\~`\~ 	
        }
    \makeatother

    \let\OriginalVerbatim=\Verbatim
    \makeatletter
    \renewcommand{\Verbatim}[1][1]{%
        %\parskip\z@skip
        \sbox\Wrappedcontinuationbox {\Wrappedcontinuationsymbol}%
        \sbox\Wrappedvisiblespacebox {\FV@SetupFont\Wrappedvisiblespace}%
        \def\FancyVerbFormatLine ##1{\hsize\linewidth
            \vtop{\raggedright\hyphenpenalty\z@\exhyphenpenalty\z@
                \doublehyphendemerits\z@\finalhyphendemerits\z@
                \strut ##1\strut}%
        }%
        % If the linebreak is at a space, the latter will be displayed as visible
        % space at end of first line, and a continuation symbol starts next line.
        % Stretch/shrink are however usually zero for typewriter font.
        \def\FV@Space {%
            \nobreak\hskip\z@ plus\fontdimen3\font minus\fontdimen4\font
            \discretionary{\copy\Wrappedvisiblespacebox}{\Wrappedafterbreak}
            {\kern\fontdimen2\font}%
        }%
        
        % Allow breaks at special characters using \PYG... macros.
        \Wrappedbreaksatspecials
        % Breaks at punctuation characters . , ; ? ! and / need catcode=\active 	
        \OriginalVerbatim[#1,codes*=\Wrappedbreaksatpunct]%
    }
    \makeatother

    % Exact colors from NB
    \definecolor{incolor}{HTML}{303F9F}
    \definecolor{outcolor}{HTML}{D84315}
    \definecolor{cellborder}{HTML}{CFCFCF}
    \definecolor{cellbackground}{HTML}{F7F7F7}
    
    % prompt
    \makeatletter
    \newcommand{\boxspacing}{\kern\kvtcb@left@rule\kern\kvtcb@boxsep}
    \makeatother
    \newcommand{\prompt}[4]{
        {\ttfamily\llap{{\color{#2}[#3]:\hspace{3pt}#4}}\vspace{-\baselineskip}}
    }
    

    
    % Prevent overflowing lines due to hard-to-break entities
    \sloppy 
    % Setup hyperref package
    \hypersetup{
      breaklinks=true,  % so long urls are correctly broken across lines
      colorlinks=true,
      urlcolor=urlcolor,
      linkcolor=linkcolor,
      citecolor=citecolor,
      }
    % Slightly bigger margins than the latex defaults
    
    \geometry{verbose,tmargin=1in,bmargin=1in,lmargin=1in,rmargin=1in}
    
    

\begin{document}
    
    \maketitle
    
    

    
    

    \begin{tcolorbox}[breakable, size=fbox, boxrule=1pt, pad at break*=1mm,colback=cellbackground, colframe=cellborder]
\prompt{In}{incolor}{ }{\boxspacing}
\begin{Verbatim}[commandchars=\\\{\}]
\PY{k+kn}{from} \PY{n+nn}{google}\PY{n+nn}{.}\PY{n+nn}{colab} \PY{k+kn}{import} \PY{n}{drive}

\PY{n}{drive}\PY{o}{.}\PY{n}{mount}\PY{p}{(}\PY{l+s+s1}{\PYZsq{}}\PY{l+s+s1}{/content/drive}\PY{l+s+s1}{\PYZsq{}}\PY{p}{,} \PY{n}{force\PYZus{}remount}\PY{o}{=}\PY{k+kc}{True}\PY{p}{)}

\PY{c+c1}{\PYZsh{} 输入daseCV所在的路径}
\PY{c+c1}{\PYZsh{} \PYZsq{}daseCV\PYZsq{} 文件夹包括 \PYZsq{}.py\PYZsq{}, \PYZsq{}classifiers\PYZsq{} 和\PYZsq{}datasets\PYZsq{}文件夹}
\PY{c+c1}{\PYZsh{} 例如 \PYZsq{}CV/assignments/assignment1/daseCV/\PYZsq{}}
\PY{n}{FOLDERNAME} \PY{o}{=} \PY{l+s+s1}{\PYZsq{}}\PY{l+s+s1}{CV/assignments/assignment1/daseCV/}\PY{l+s+s1}{\PYZsq{}}

\PY{k}{assert} \PY{n}{FOLDERNAME} \PY{o+ow}{is} \PY{o+ow}{not} \PY{k+kc}{None}\PY{p}{,} \PY{l+s+s2}{\PYZdq{}}\PY{l+s+s2}{[!] Enter the foldername.}\PY{l+s+s2}{\PYZdq{}}

\PY{o}{\PYZpc{}}\PY{k}{cd} drive/My\PYZbs{} Drive
\PY{o}{\PYZpc{}}\PY{k}{cp} \PYZhy{}r \PYZdl{}FOLDERNAME ../../
\PY{o}{\PYZpc{}}\PY{k}{cd} ../../
\PY{o}{\PYZpc{}}\PY{k}{cd} daseCV/datasets/
\PY{o}{!}bash get\PYZus{}datasets.sh
\PY{o}{\PYZpc{}}\PY{k}{cd} ../../
\end{Verbatim}
\end{tcolorbox}

    \begin{Verbatim}[commandchars=\\\{\}]
Mounted at /content/drive
/content/drive/My Drive
/content
/content/daseCV/datasets
--2021-04-03 09:01:31--  http://www.cs.toronto.edu/\textasciitilde{}kriz/cifar-10-python.tar.gz
Resolving www.cs.toronto.edu (www.cs.toronto.edu){\ldots} 128.100.3.30
Connecting to www.cs.toronto.edu (www.cs.toronto.edu)|128.100.3.30|:80{\ldots}
connected.
HTTP request sent, awaiting response{\ldots} 200 OK
Length: 170498071 (163M) [application/x-gzip]
Saving to: ‘cifar-10-python.tar.gz’

cifar-10-python.tar 100\%[===================>] 162.60M  45.7MB/s    in 4.0s

2021-04-03 09:01:35 (40.9 MB/s) - ‘cifar-10-python.tar.gz’ saved
[170498071/170498071]

cifar-10-batches-py/
cifar-10-batches-py/data\_batch\_4
cifar-10-batches-py/readme.html
cifar-10-batches-py/test\_batch
cifar-10-batches-py/data\_batch\_3
cifar-10-batches-py/batches.meta
cifar-10-batches-py/data\_batch\_2
cifar-10-batches-py/data\_batch\_5
cifar-10-batches-py/data\_batch\_1
/content
    \end{Verbatim}

    \hypertarget{k-ux8fd1ux90bbux7b97ux6cd5-knn-ux7ec3ux4e60}{%
\section{K-近邻算法 (kNN)
练习}\label{k-ux8fd1ux90bbux7b97ux6cd5-knn-ux7ec3ux4e60}}

\emph{补充并完成本练习。}

kNN分类器包含两个阶段:

\begin{itemize}
\tightlist
\item
  训练阶段,分类器获取训练数据并简单地记住它。
\item
  测试阶段,
  kNN将测试图像与所有训练图像进行比较,并计算出前k个最相似的训练示例的标签来对每个测试图像进行分类。
\item
  对k值进行交叉验证
\end{itemize}

在本练习中,您将实现这些步骤,并了解基本的图像分类、交叉验证和熟练编写高效矢量化代码的能力。

    \begin{tcolorbox}[breakable, size=fbox, boxrule=1pt, pad at break*=1mm,colback=cellbackground, colframe=cellborder]
\prompt{In}{incolor}{ }{\boxspacing}
\begin{Verbatim}[commandchars=\\\{\}]
\PY{c+c1}{\PYZsh{} 运行notebook的一些初始化代码}

\PY{k+kn}{import} \PY{n+nn}{random}
\PY{k+kn}{import} \PY{n+nn}{numpy} \PY{k}{as} \PY{n+nn}{np}
\PY{k+kn}{from} \PY{n+nn}{daseCV}\PY{n+nn}{.}\PY{n+nn}{data\PYZus{}utils} \PY{k+kn}{import} \PY{n}{load\PYZus{}CIFAR10}
\PY{k+kn}{import} \PY{n+nn}{matplotlib}\PY{n+nn}{.}\PY{n+nn}{pyplot} \PY{k}{as} \PY{n+nn}{plt}

\PY{c+c1}{\PYZsh{} 使得matplotlib的图像在当前页显示而不是新的窗口。}
\PY{o}{\PYZpc{}}\PY{k}{matplotlib} inline
\PY{n}{plt}\PY{o}{.}\PY{n}{rcParams}\PY{p}{[}\PY{l+s+s1}{\PYZsq{}}\PY{l+s+s1}{figure.figsize}\PY{l+s+s1}{\PYZsq{}}\PY{p}{]} \PY{o}{=} \PY{p}{(}\PY{l+m+mf}{10.0}\PY{p}{,} \PY{l+m+mf}{8.0}\PY{p}{)} \PY{c+c1}{\PYZsh{} set default size of plots}
\PY{n}{plt}\PY{o}{.}\PY{n}{rcParams}\PY{p}{[}\PY{l+s+s1}{\PYZsq{}}\PY{l+s+s1}{image.interpolation}\PY{l+s+s1}{\PYZsq{}}\PY{p}{]} \PY{o}{=} \PY{l+s+s1}{\PYZsq{}}\PY{l+s+s1}{nearest}\PY{l+s+s1}{\PYZsq{}}
\PY{n}{plt}\PY{o}{.}\PY{n}{rcParams}\PY{p}{[}\PY{l+s+s1}{\PYZsq{}}\PY{l+s+s1}{image.cmap}\PY{l+s+s1}{\PYZsq{}}\PY{p}{]} \PY{o}{=} \PY{l+s+s1}{\PYZsq{}}\PY{l+s+s1}{gray}\PY{l+s+s1}{\PYZsq{}}

\PY{c+c1}{\PYZsh{} 一些更神奇的,使notebook重新加载外部的python模块;}
\PY{c+c1}{\PYZsh{} 参见 http://stackoverflow.com/questions/1907993/autoreload\PYZhy{}of\PYZhy{}modules\PYZhy{}in\PYZhy{}ipython}
\PY{o}{\PYZpc{}}\PY{k}{load\PYZus{}ext} autoreload
\PY{o}{\PYZpc{}}\PY{k}{autoreload} 2
\end{Verbatim}
\end{tcolorbox}

    \begin{tcolorbox}[breakable, size=fbox, boxrule=1pt, pad at break*=1mm,colback=cellbackground, colframe=cellborder]
\prompt{In}{incolor}{ }{\boxspacing}
\begin{Verbatim}[commandchars=\\\{\}]
\PY{c+c1}{\PYZsh{} 加载未处理的 CIFAR\PYZhy{}10 数据.}
\PY{n}{cifar10\PYZus{}dir} \PY{o}{=} \PY{l+s+s1}{\PYZsq{}}\PY{l+s+s1}{daseCV/datasets/cifar\PYZhy{}10\PYZhy{}batches\PYZhy{}py}\PY{l+s+s1}{\PYZsq{}}

\PY{c+c1}{\PYZsh{} 清理变量以防止多次加载数据(这可能会导致内存问题)}
\PY{k}{try}\PY{p}{:}
   \PY{k}{del} \PY{n}{X\PYZus{}train}\PY{p}{,} \PY{n}{y\PYZus{}train}
   \PY{k}{del} \PY{n}{X\PYZus{}test}\PY{p}{,} \PY{n}{y\PYZus{}test}
   \PY{n+nb}{print}\PY{p}{(}\PY{l+s+s1}{\PYZsq{}}\PY{l+s+s1}{Clear previously loaded data.}\PY{l+s+s1}{\PYZsq{}}\PY{p}{)}
\PY{k}{except}\PY{p}{:}
   \PY{k}{pass}

\PY{n}{X\PYZus{}train}\PY{p}{,} \PY{n}{y\PYZus{}train}\PY{p}{,} \PY{n}{X\PYZus{}test}\PY{p}{,} \PY{n}{y\PYZus{}test} \PY{o}{=} \PY{n}{load\PYZus{}CIFAR10}\PY{p}{(}\PY{n}{cifar10\PYZus{}dir}\PY{p}{)}

\PY{c+c1}{\PYZsh{} 作为健全性检查,我们打印出训练和测试数据的形状。}
\PY{n+nb}{print}\PY{p}{(}\PY{l+s+s1}{\PYZsq{}}\PY{l+s+s1}{Training data shape: }\PY{l+s+s1}{\PYZsq{}}\PY{p}{,} \PY{n}{X\PYZus{}train}\PY{o}{.}\PY{n}{shape}\PY{p}{)}
\PY{n+nb}{print}\PY{p}{(}\PY{l+s+s1}{\PYZsq{}}\PY{l+s+s1}{Training labels shape: }\PY{l+s+s1}{\PYZsq{}}\PY{p}{,} \PY{n}{y\PYZus{}train}\PY{o}{.}\PY{n}{shape}\PY{p}{)}
\PY{n+nb}{print}\PY{p}{(}\PY{l+s+s1}{\PYZsq{}}\PY{l+s+s1}{Test data shape: }\PY{l+s+s1}{\PYZsq{}}\PY{p}{,} \PY{n}{X\PYZus{}test}\PY{o}{.}\PY{n}{shape}\PY{p}{)}
\PY{n+nb}{print}\PY{p}{(}\PY{l+s+s1}{\PYZsq{}}\PY{l+s+s1}{Test labels shape: }\PY{l+s+s1}{\PYZsq{}}\PY{p}{,} \PY{n}{y\PYZus{}test}\PY{o}{.}\PY{n}{shape}\PY{p}{)}
\end{Verbatim}
\end{tcolorbox}

    \begin{Verbatim}[commandchars=\\\{\}]
Training data shape:  (50000, 32, 32, 3)
Training labels shape:  (50000,)
Test data shape:  (10000, 32, 32, 3)
Test labels shape:  (10000,)
    \end{Verbatim}

    \begin{tcolorbox}[breakable, size=fbox, boxrule=1pt, pad at break*=1mm,colback=cellbackground, colframe=cellborder]
\prompt{In}{incolor}{ }{\boxspacing}
\begin{Verbatim}[commandchars=\\\{\}]
\PY{c+c1}{\PYZsh{} 可视化数据集中的一些示例。}
\PY{c+c1}{\PYZsh{} 我们展示了训练图像的所有类别的一些示例。}
\PY{n}{classes} \PY{o}{=} \PY{p}{[}\PY{l+s+s1}{\PYZsq{}}\PY{l+s+s1}{plane}\PY{l+s+s1}{\PYZsq{}}\PY{p}{,} \PY{l+s+s1}{\PYZsq{}}\PY{l+s+s1}{car}\PY{l+s+s1}{\PYZsq{}}\PY{p}{,} \PY{l+s+s1}{\PYZsq{}}\PY{l+s+s1}{bird}\PY{l+s+s1}{\PYZsq{}}\PY{p}{,} \PY{l+s+s1}{\PYZsq{}}\PY{l+s+s1}{cat}\PY{l+s+s1}{\PYZsq{}}\PY{p}{,} \PY{l+s+s1}{\PYZsq{}}\PY{l+s+s1}{deer}\PY{l+s+s1}{\PYZsq{}}\PY{p}{,} \PY{l+s+s1}{\PYZsq{}}\PY{l+s+s1}{dog}\PY{l+s+s1}{\PYZsq{}}\PY{p}{,} \PY{l+s+s1}{\PYZsq{}}\PY{l+s+s1}{frog}\PY{l+s+s1}{\PYZsq{}}\PY{p}{,} \PY{l+s+s1}{\PYZsq{}}\PY{l+s+s1}{horse}\PY{l+s+s1}{\PYZsq{}}\PY{p}{,} \PY{l+s+s1}{\PYZsq{}}\PY{l+s+s1}{ship}\PY{l+s+s1}{\PYZsq{}}\PY{p}{,} \PY{l+s+s1}{\PYZsq{}}\PY{l+s+s1}{truck}\PY{l+s+s1}{\PYZsq{}}\PY{p}{]}
\PY{n}{num\PYZus{}classes} \PY{o}{=} \PY{n+nb}{len}\PY{p}{(}\PY{n}{classes}\PY{p}{)}
\PY{n}{samples\PYZus{}per\PYZus{}class} \PY{o}{=} \PY{l+m+mi}{7}
\PY{k}{for} \PY{n}{y}\PY{p}{,} \PY{n+nb+bp}{cls} \PY{o+ow}{in} \PY{n+nb}{enumerate}\PY{p}{(}\PY{n}{classes}\PY{p}{)}\PY{p}{:}
    \PY{n}{idxs} \PY{o}{=} \PY{n}{np}\PY{o}{.}\PY{n}{flatnonzero}\PY{p}{(}\PY{n}{y\PYZus{}train} \PY{o}{==} \PY{n}{y}\PY{p}{)} \PY{c+c1}{\PYZsh{} flatnonzero表示返回所给数列的非零项的索引值,这里表示返回所有属于y类的索引}
    \PY{n}{idxs} \PY{o}{=} \PY{n}{np}\PY{o}{.}\PY{n}{random}\PY{o}{.}\PY{n}{choice}\PY{p}{(}\PY{n}{idxs}\PY{p}{,} \PY{n}{samples\PYZus{}per\PYZus{}class}\PY{p}{,} \PY{n}{replace}\PY{o}{=}\PY{k+kc}{False}\PY{p}{)} \PY{c+c1}{\PYZsh{} replace表示抽取的样本是否能重复}
    \PY{k}{for} \PY{n}{i}\PY{p}{,} \PY{n}{idx} \PY{o+ow}{in} \PY{n+nb}{enumerate}\PY{p}{(}\PY{n}{idxs}\PY{p}{)}\PY{p}{:}
        \PY{n}{plt\PYZus{}idx} \PY{o}{=} \PY{n}{i} \PY{o}{*} \PY{n}{num\PYZus{}classes} \PY{o}{+} \PY{n}{y} \PY{o}{+} \PY{l+m+mi}{1}
        \PY{n}{plt}\PY{o}{.}\PY{n}{subplot}\PY{p}{(}\PY{n}{samples\PYZus{}per\PYZus{}class}\PY{p}{,} \PY{n}{num\PYZus{}classes}\PY{p}{,} \PY{n}{plt\PYZus{}idx}\PY{p}{)}
        \PY{n}{plt}\PY{o}{.}\PY{n}{imshow}\PY{p}{(}\PY{n}{X\PYZus{}train}\PY{p}{[}\PY{n}{idx}\PY{p}{]}\PY{o}{.}\PY{n}{astype}\PY{p}{(}\PY{l+s+s1}{\PYZsq{}}\PY{l+s+s1}{uint8}\PY{l+s+s1}{\PYZsq{}}\PY{p}{)}\PY{p}{)}
        \PY{n}{plt}\PY{o}{.}\PY{n}{axis}\PY{p}{(}\PY{l+s+s1}{\PYZsq{}}\PY{l+s+s1}{off}\PY{l+s+s1}{\PYZsq{}}\PY{p}{)}
        \PY{k}{if} \PY{n}{i} \PY{o}{==} \PY{l+m+mi}{0}\PY{p}{:}
            \PY{n}{plt}\PY{o}{.}\PY{n}{title}\PY{p}{(}\PY{n+nb+bp}{cls}\PY{p}{)}
\PY{n}{plt}\PY{o}{.}\PY{n}{show}\PY{p}{(}\PY{p}{)}
\end{Verbatim}
\end{tcolorbox}

    \begin{center}
    \adjustimage{max size={0.9\linewidth}{0.9\paperheight}}{output_5_0.png}
    \end{center}
    { \hspace*{\fill} \\}
    
    \begin{tcolorbox}[breakable, size=fbox, boxrule=1pt, pad at break*=1mm,colback=cellbackground, colframe=cellborder]
\prompt{In}{incolor}{ }{\boxspacing}
\begin{Verbatim}[commandchars=\\\{\}]
\PY{c+c1}{\PYZsh{} 在练习中使用更小的子样本可以提高代码的效率}
\PY{n}{num\PYZus{}training} \PY{o}{=} \PY{l+m+mi}{5000}
\PY{n}{mask} \PY{o}{=} \PY{n+nb}{list}\PY{p}{(}\PY{n+nb}{range}\PY{p}{(}\PY{n}{num\PYZus{}training}\PY{p}{)}\PY{p}{)}
\PY{n}{X\PYZus{}train} \PY{o}{=} \PY{n}{X\PYZus{}train}\PY{p}{[}\PY{n}{mask}\PY{p}{]}
\PY{n}{y\PYZus{}train} \PY{o}{=} \PY{n}{y\PYZus{}train}\PY{p}{[}\PY{n}{mask}\PY{p}{]}

\PY{n}{num\PYZus{}test} \PY{o}{=} \PY{l+m+mi}{500}
\PY{n}{mask} \PY{o}{=} \PY{n+nb}{list}\PY{p}{(}\PY{n+nb}{range}\PY{p}{(}\PY{n}{num\PYZus{}test}\PY{p}{)}\PY{p}{)}
\PY{n}{X\PYZus{}test} \PY{o}{=} \PY{n}{X\PYZus{}test}\PY{p}{[}\PY{n}{mask}\PY{p}{]}
\PY{n}{y\PYZus{}test} \PY{o}{=} \PY{n}{y\PYZus{}test}\PY{p}{[}\PY{n}{mask}\PY{p}{]}

\PY{c+c1}{\PYZsh{} 将图像数据调整为行}
\PY{n}{X\PYZus{}train} \PY{o}{=} \PY{n}{np}\PY{o}{.}\PY{n}{reshape}\PY{p}{(}\PY{n}{X\PYZus{}train}\PY{p}{,} \PY{p}{(}\PY{n}{X\PYZus{}train}\PY{o}{.}\PY{n}{shape}\PY{p}{[}\PY{l+m+mi}{0}\PY{p}{]}\PY{p}{,} \PY{o}{\PYZhy{}}\PY{l+m+mi}{1}\PY{p}{)}\PY{p}{)}
\PY{n}{X\PYZus{}test} \PY{o}{=} \PY{n}{np}\PY{o}{.}\PY{n}{reshape}\PY{p}{(}\PY{n}{X\PYZus{}test}\PY{p}{,} \PY{p}{(}\PY{n}{X\PYZus{}test}\PY{o}{.}\PY{n}{shape}\PY{p}{[}\PY{l+m+mi}{0}\PY{p}{]}\PY{p}{,} \PY{o}{\PYZhy{}}\PY{l+m+mi}{1}\PY{p}{)}\PY{p}{)}
\PY{n+nb}{print}\PY{p}{(}\PY{n}{X\PYZus{}train}\PY{o}{.}\PY{n}{shape}\PY{p}{,} \PY{n}{X\PYZus{}test}\PY{o}{.}\PY{n}{shape}\PY{p}{)}
\end{Verbatim}
\end{tcolorbox}

    \begin{Verbatim}[commandchars=\\\{\}]
(5000, 3072) (500, 3072)
    \end{Verbatim}

    \begin{tcolorbox}[breakable, size=fbox, boxrule=1pt, pad at break*=1mm,colback=cellbackground, colframe=cellborder]
\prompt{In}{incolor}{ }{\boxspacing}
\begin{Verbatim}[commandchars=\\\{\}]
\PY{n}{X\PYZus{}test}\PY{p}{[}\PY{l+m+mi}{1}\PY{p}{:}\PY{l+m+mi}{3}\PY{p}{,}\PY{l+m+mi}{2}\PY{p}{:}\PY{l+m+mi}{6}\PY{p}{]}\PY{p}{[}\PY{l+m+mi}{1}\PY{p}{,}\PY{p}{:}\PY{p}{]}
\end{Verbatim}
\end{tcolorbox}

            \begin{tcolorbox}[breakable, size=fbox, boxrule=.5pt, pad at break*=1mm, opacityfill=0]
\prompt{Out}{outcolor}{ }{\boxspacing}
\begin{Verbatim}[commandchars=\\\{\}]
array([222., 158., 187., 218.])
\end{Verbatim}
\end{tcolorbox}
        
    \begin{tcolorbox}[breakable, size=fbox, boxrule=1pt, pad at break*=1mm,colback=cellbackground, colframe=cellborder]
\prompt{In}{incolor}{ }{\boxspacing}
\begin{Verbatim}[commandchars=\\\{\}]
\PY{n}{np}\PY{o}{.}\PY{n}{sqrt}\PY{p}{(}\PY{n+nb}{sum}\PY{p}{(}\PY{p}{[}\PY{l+m+mi}{3}\PY{p}{,}\PY{l+m+mi}{6}\PY{p}{]}\PY{p}{)}\PY{p}{)}
\end{Verbatim}
\end{tcolorbox}

            \begin{tcolorbox}[breakable, size=fbox, boxrule=.5pt, pad at break*=1mm, opacityfill=0]
\prompt{Out}{outcolor}{ }{\boxspacing}
\begin{Verbatim}[commandchars=\\\{\}]
3.0
\end{Verbatim}
\end{tcolorbox}
        
    \begin{tcolorbox}[breakable, size=fbox, boxrule=1pt, pad at break*=1mm,colback=cellbackground, colframe=cellborder]
\prompt{In}{incolor}{ }{\boxspacing}
\begin{Verbatim}[commandchars=\\\{\}]
\PY{p}{[}\PY{n}{i} \PY{k}{for} \PY{n}{i} \PY{o+ow}{in} \PY{n+nb}{range}\PY{p}{(}\PY{l+m+mi}{3}\PY{p}{)}\PY{p}{]}
\end{Verbatim}
\end{tcolorbox}

            \begin{tcolorbox}[breakable, size=fbox, boxrule=.5pt, pad at break*=1mm, opacityfill=0]
\prompt{Out}{outcolor}{ }{\boxspacing}
\begin{Verbatim}[commandchars=\\\{\}]
[0, 1, 2]
\end{Verbatim}
\end{tcolorbox}
        
    \begin{tcolorbox}[breakable, size=fbox, boxrule=1pt, pad at break*=1mm,colback=cellbackground, colframe=cellborder]
\prompt{In}{incolor}{ }{\boxspacing}
\begin{Verbatim}[commandchars=\\\{\}]
\PY{k+kn}{from} \PY{n+nn}{daseCV}\PY{n+nn}{.}\PY{n+nn}{classifiers} \PY{k+kn}{import} \PY{n}{KNearestNeighbor}

\PY{c+c1}{\PYZsh{} 创建一个kNN分类器实例。}
\PY{c+c1}{\PYZsh{} 请记住,kNN分类器的训练并不会做什么: }
\PY{c+c1}{\PYZsh{} 分类器仅记住数据并且不做进一步处理}
\PY{n}{classifier} \PY{o}{=} \PY{n}{KNearestNeighbor}\PY{p}{(}\PY{p}{)}
\PY{n}{classifier}\PY{o}{.}\PY{n}{train}\PY{p}{(}\PY{n}{X\PYZus{}train}\PY{p}{,} \PY{n}{y\PYZus{}train}\PY{p}{)}
\end{Verbatim}
\end{tcolorbox}

    现在,我们要使用kNN分类器对测试数据进行分类。回想一下,我们可以将该过程分为两个步骤:

\begin{enumerate}
\def\labelenumi{\arabic{enumi}.}
\tightlist
\item
  首先,我们必须计算所有测试样本与所有训练样本之间的距离。
\item
  给定这些距离,对于每个测试示例,我们找到k个最接近的示例,并让它们对标签进行投票
\end{enumerate}

让我们开始计算所有训练和测试示例之间的距离矩阵。 假设有 \textbf{Ntr}
的训练样本和 \textbf{Nte} 的测试样本, 该过程的结果存储在一个 \textbf{Nte
x Ntr} 矩阵中,其中每个元素 (i,j) 表示的是第 i 个测试样本和第 j 个
训练样本的距离。

\textbf{注意:
在完成此notebook中的三个距离的计算时请不要使用numpy提供的np.linalg.norm()函数。}

首先打开 \texttt{daseCV/classifiers/k\_nearest\_neighbor.py}
并且补充完成函数 \texttt{compute\_distances\_two\_loops}
,这个函数使用双重循环(效率十分低下)来计算距离矩阵。

    \begin{tcolorbox}[breakable, size=fbox, boxrule=1pt, pad at break*=1mm,colback=cellbackground, colframe=cellborder]
\prompt{In}{incolor}{ }{\boxspacing}
\begin{Verbatim}[commandchars=\\\{\}]
\PY{n}{X\PYZus{}train}\PY{p}{[}\PY{l+m+mi}{0}\PY{p}{,}\PY{l+m+mi}{2}\PY{p}{:}\PY{l+m+mi}{5}\PY{p}{]}\PY{o}{*}\PY{o}{*}\PY{l+m+mi}{2}\PY{o}{+}\PY{n}{X\PYZus{}train}\PY{p}{[}\PY{l+m+mi}{1}\PY{p}{,}\PY{l+m+mi}{2}\PY{p}{:}\PY{l+m+mi}{5}\PY{p}{]}\PY{o}{*}\PY{o}{*}\PY{l+m+mi}{2}
\end{Verbatim}
\end{tcolorbox}

            \begin{tcolorbox}[breakable, size=fbox, boxrule=.5pt, pad at break*=1mm, opacityfill=0]
\prompt{Out}{outcolor}{ }{\boxspacing}
\begin{Verbatim}[commandchars=\\\{\}]
array([38938., 17725., 20885.])
\end{Verbatim}
\end{tcolorbox}
        
    \begin{tcolorbox}[breakable, size=fbox, boxrule=1pt, pad at break*=1mm,colback=cellbackground, colframe=cellborder]
\prompt{In}{incolor}{ }{\boxspacing}
\begin{Verbatim}[commandchars=\\\{\}]
\PY{c+c1}{\PYZsh{} 打开 daseCV/classifiers/k\PYZus{}nearest\PYZus{}neighbor.py 并且补充完成}
\PY{c+c1}{\PYZsh{} compute\PYZus{}distances\PYZus{}two\PYZus{}loops.}

\PY{c+c1}{\PYZsh{} 测试你的代码:}
\PY{n}{dists} \PY{o}{=} \PY{n}{classifier}\PY{o}{.}\PY{n}{compute\PYZus{}distances\PYZus{}two\PYZus{}loops}\PY{p}{(}\PY{n}{X\PYZus{}test}\PY{p}{)}
\PY{n+nb}{print}\PY{p}{(}\PY{n}{dists}\PY{o}{.}\PY{n}{shape}\PY{p}{)}
\end{Verbatim}
\end{tcolorbox}

    \begin{Verbatim}[commandchars=\\\{\}]
(500, 5000)
    \end{Verbatim}

    \begin{tcolorbox}[breakable, size=fbox, boxrule=1pt, pad at break*=1mm,colback=cellbackground, colframe=cellborder]
\prompt{In}{incolor}{ }{\boxspacing}
\begin{Verbatim}[commandchars=\\\{\}]
\PY{c+c1}{\PYZsh{} 我们可视化距离矩阵:每行代表一个测试样本与训练样本的距离}
\PY{n}{plt}\PY{o}{.}\PY{n}{imshow}\PY{p}{(}\PY{n}{dists}\PY{p}{,} \PY{n}{interpolation}\PY{o}{=}\PY{l+s+s1}{\PYZsq{}}\PY{l+s+s1}{none}\PY{l+s+s1}{\PYZsq{}}\PY{p}{)}
\PY{n}{plt}\PY{o}{.}\PY{n}{show}\PY{p}{(}\PY{p}{)}
\end{Verbatim}
\end{tcolorbox}

    \begin{center}
    \adjustimage{max size={0.9\linewidth}{0.9\paperheight}}{output_14_0.png}
    \end{center}
    { \hspace*{\fill} \\}
    
    \begin{tcolorbox}[breakable, size=fbox, boxrule=1pt, pad at break*=1mm,colback=cellbackground, colframe=cellborder]
\prompt{In}{incolor}{ }{\boxspacing}
\begin{Verbatim}[commandchars=\\\{\}]

\end{Verbatim}
\end{tcolorbox}

    \textbf{问题 1}

请注意距离矩阵中的结构化图案,其中某些行或列的可见亮度更高。(请注意,使用默认的配色方案,黑色表示低距离,而白色表示高距离。)

\begin{itemize}
\tightlist
\item
  数据中导致行亮度更高的原因是什么?
\item
  那列方向的是什么原因呢?
\end{itemize}

\(\color{blue}{\textit 答:}\)
对行:某测试数据与原数据距离越近,越相似,亮度越低。对列:某训练数据与测试数据距离越近,越相似,亮度越低。

    \begin{tcolorbox}[breakable, size=fbox, boxrule=1pt, pad at break*=1mm,colback=cellbackground, colframe=cellborder]
\prompt{In}{incolor}{ }{\boxspacing}
\begin{Verbatim}[commandchars=\\\{\}]
\PY{c+c1}{\PYZsh{} 现在实现函数predict\PYZus{}labels并运行以下代码:}
\PY{c+c1}{\PYZsh{} 我们使用k = 1(这是最近的邻居)。}
\PY{n}{y\PYZus{}test\PYZus{}pred} \PY{o}{=} \PY{n}{classifier}\PY{o}{.}\PY{n}{predict\PYZus{}labels}\PY{p}{(}\PY{n}{dists}\PY{p}{,} \PY{n}{k}\PY{o}{=}\PY{l+m+mi}{1}\PY{p}{)}

\PY{c+c1}{\PYZsh{} 计算并打印出预测的精度}
\PY{n}{num\PYZus{}correct} \PY{o}{=} \PY{n}{np}\PY{o}{.}\PY{n}{sum}\PY{p}{(}\PY{n}{y\PYZus{}test\PYZus{}pred} \PY{o}{==} \PY{n}{y\PYZus{}test}\PY{p}{)}
\PY{n}{accuracy} \PY{o}{=} \PY{n+nb}{float}\PY{p}{(}\PY{n}{num\PYZus{}correct}\PY{p}{)} \PY{o}{/} \PY{n}{num\PYZus{}test}
\PY{n+nb}{print}\PY{p}{(}\PY{l+s+s1}{\PYZsq{}}\PY{l+s+s1}{Got }\PY{l+s+si}{\PYZpc{}d}\PY{l+s+s1}{ / }\PY{l+s+si}{\PYZpc{}d}\PY{l+s+s1}{ correct =\PYZgt{} accuracy: }\PY{l+s+si}{\PYZpc{}f}\PY{l+s+s1}{\PYZsq{}} \PY{o}{\PYZpc{}} \PY{p}{(}\PY{n}{num\PYZus{}correct}\PY{p}{,} \PY{n}{num\PYZus{}test}\PY{p}{,} \PY{n}{accuracy}\PY{p}{)}\PY{p}{)}
\end{Verbatim}
\end{tcolorbox}

    \begin{Verbatim}[commandchars=\\\{\}]
Got 137 / 500 correct => accuracy: 0.274000
    \end{Verbatim}

    你预期的精度应该为 \texttt{27\%} 左右。 现在让我们尝试更大的 \texttt{k},
比如 \texttt{k\ =\ 5}:

    \begin{tcolorbox}[breakable, size=fbox, boxrule=1pt, pad at break*=1mm,colback=cellbackground, colframe=cellborder]
\prompt{In}{incolor}{ }{\boxspacing}
\begin{Verbatim}[commandchars=\\\{\}]
\PY{n}{y\PYZus{}test\PYZus{}pred} \PY{o}{=} \PY{n}{classifier}\PY{o}{.}\PY{n}{predict\PYZus{}labels}\PY{p}{(}\PY{n}{dists}\PY{p}{,} \PY{n}{k}\PY{o}{=}\PY{l+m+mi}{5}\PY{p}{)}
\PY{n}{num\PYZus{}correct} \PY{o}{=} \PY{n}{np}\PY{o}{.}\PY{n}{sum}\PY{p}{(}\PY{n}{y\PYZus{}test\PYZus{}pred} \PY{o}{==} \PY{n}{y\PYZus{}test}\PY{p}{)}
\PY{n}{accuracy} \PY{o}{=} \PY{n+nb}{float}\PY{p}{(}\PY{n}{num\PYZus{}correct}\PY{p}{)} \PY{o}{/} \PY{n}{num\PYZus{}test}
\PY{n+nb}{print}\PY{p}{(}\PY{l+s+s1}{\PYZsq{}}\PY{l+s+s1}{Got }\PY{l+s+si}{\PYZpc{}d}\PY{l+s+s1}{ / }\PY{l+s+si}{\PYZpc{}d}\PY{l+s+s1}{ correct =\PYZgt{} accuracy: }\PY{l+s+si}{\PYZpc{}f}\PY{l+s+s1}{\PYZsq{}} \PY{o}{\PYZpc{}} \PY{p}{(}\PY{n}{num\PYZus{}correct}\PY{p}{,} \PY{n}{num\PYZus{}test}\PY{p}{,} \PY{n}{accuracy}\PY{p}{)}\PY{p}{)}
\end{Verbatim}
\end{tcolorbox}

    \begin{Verbatim}[commandchars=\\\{\}]
Got 139 / 500 correct => accuracy: 0.278000
    \end{Verbatim}

    你应该能看到一个比 \texttt{k\ =\ 1} 稍微好一点的结果。

    \textbf{问题 2}

我们还可以使用其他距离指标,例如L1距离。

记图像 \(I_k\) 的每个位置 \((i,j)\) 的像素值为 \(p_{ij}^{(k)}\),

所有图像上的所有像素的均值 \(\mu\) 为

\[\mu=\frac{1}{nhw}\sum_{k=1}^n\sum_{i=1}^{h}\sum_{j=1}^{w}p_{ij}^{(k)}\]

并且所有图像的每个像素的均值 \(\mu_{ij}\) 为

\[\mu_{ij}=\frac{1}{n}\sum_{k=1}^np_{ij}^{(k)}.\]

标准差 \(\sigma\) 以及每个像素的标准差 \(\sigma_{ij}\) 的定义与之类似。

以下哪个预处理步骤不会改变使用L1距离的最近邻分类器的效果?选择所有符合条件的答案。
1. 减去均值 \(\mu\) (\(\tilde{p}_{ij}^{(k)}=p_{ij}^{(k)}-\mu\).) 2.
减去每个像素均值 \(\mu_{ij}\)
(\(\tilde{p}_{ij}^{(k)}=p_{ij}^{(k)}-\mu_{ij}\).) 3. 减去均值 \(\mu\)
然后除以标准偏差 \(\sigma\). 4. 减去每个像素均值 \(\mu_{ij}\)
并除以每个素标准差 \(\sigma_{ij}\). 5. 旋转数据的坐标轴。

\(\color{blue}{\textit 你的回答:}\) 13

\(\color{blue}{\textit 你的解释:}\)
1,问的是分类效果,而平移不影响相对位置和临界线,所以没影响
2,图像与图像的距离会变化 3,相当于放缩,对分类效果没影响
4,图像与图像的距离会变化 5,对于L2距离而言不变,对L1距离不符合

    \begin{tcolorbox}[breakable, size=fbox, boxrule=1pt, pad at break*=1mm,colback=cellbackground, colframe=cellborder]
\prompt{In}{incolor}{ }{\boxspacing}
\begin{Verbatim}[commandchars=\\\{\}]
\PY{c+c1}{\PYZsh{} 现在,通过部分矢量化并且使用单层循环的来加快距离矩阵的计算。}
\PY{c+c1}{\PYZsh{} 需要实现函数compute\PYZus{}distances\PYZus{}one\PYZus{}loop并运行以下代码:}

\PY{n}{dists\PYZus{}one} \PY{o}{=} \PY{n}{classifier}\PY{o}{.}\PY{n}{compute\PYZus{}distances\PYZus{}one\PYZus{}loop}\PY{p}{(}\PY{n}{X\PYZus{}test}\PY{p}{)}

\PY{c+c1}{\PYZsh{} 为了确保我们的矢量化实现正确,我们要保证它的结果与最原始的实现方式结果一致。}
\PY{c+c1}{\PYZsh{} 有很多方法可以确定两个矩阵是否相似。最简单的方法之一就是Frobenius范数。 }
\PY{c+c1}{\PYZsh{} 如果您以前从未了解过Frobenius范数,它其实是两个矩阵的所有元素之差的平方和的平方根;}
\PY{c+c1}{\PYZsh{} 换句话说,就是将矩阵重整为向量并计算它们之间的欧几里得距离。}

\PY{n}{difference} \PY{o}{=} \PY{n}{np}\PY{o}{.}\PY{n}{linalg}\PY{o}{.}\PY{n}{norm}\PY{p}{(}\PY{n}{dists} \PY{o}{\PYZhy{}} \PY{n}{dists\PYZus{}one}\PY{p}{,} \PY{n+nb}{ord}\PY{o}{=}\PY{l+s+s1}{\PYZsq{}}\PY{l+s+s1}{fro}\PY{l+s+s1}{\PYZsq{}}\PY{p}{)}
\PY{n+nb}{print}\PY{p}{(}\PY{l+s+s1}{\PYZsq{}}\PY{l+s+s1}{One loop difference was: }\PY{l+s+si}{\PYZpc{}f}\PY{l+s+s1}{\PYZsq{}} \PY{o}{\PYZpc{}} \PY{p}{(}\PY{n}{difference}\PY{p}{,} \PY{p}{)}\PY{p}{)}
\PY{k}{if} \PY{n}{difference} \PY{o}{\PYZlt{}} \PY{l+m+mf}{0.001}\PY{p}{:}
    \PY{n+nb}{print}\PY{p}{(}\PY{l+s+s1}{\PYZsq{}}\PY{l+s+s1}{Good! The distance matrices are the same}\PY{l+s+s1}{\PYZsq{}}\PY{p}{)}
\PY{k}{else}\PY{p}{:}
    \PY{n+nb}{print}\PY{p}{(}\PY{l+s+s1}{\PYZsq{}}\PY{l+s+s1}{Uh\PYZhy{}oh! The distance matrices are different}\PY{l+s+s1}{\PYZsq{}}\PY{p}{)}
\end{Verbatim}
\end{tcolorbox}

    \begin{Verbatim}[commandchars=\\\{\}]
One loop difference was: 0.000000
Good! The distance matrices are the same
    \end{Verbatim}

    \begin{tcolorbox}[breakable, size=fbox, boxrule=1pt, pad at break*=1mm,colback=cellbackground, colframe=cellborder]
\prompt{In}{incolor}{ }{\boxspacing}
\begin{Verbatim}[commandchars=\\\{\}]
\PY{c+c1}{\PYZsh{} 现在完成compute\PYZus{}distances\PYZus{}no\PYZus{}loops实现完全矢量化的版本并运行代码}
\PY{n}{dists\PYZus{}two} \PY{o}{=} \PY{n}{classifier}\PY{o}{.}\PY{n}{compute\PYZus{}distances\PYZus{}no\PYZus{}loops}\PY{p}{(}\PY{n}{X\PYZus{}test}\PY{p}{)}

\PY{c+c1}{\PYZsh{} 检查距离矩阵是否与我们之前计算出的矩阵一致:}
\PY{n}{difference} \PY{o}{=} \PY{n}{np}\PY{o}{.}\PY{n}{linalg}\PY{o}{.}\PY{n}{norm}\PY{p}{(}\PY{n}{dists} \PY{o}{\PYZhy{}} \PY{n}{dists\PYZus{}two}\PY{p}{,} \PY{n+nb}{ord}\PY{o}{=}\PY{l+s+s1}{\PYZsq{}}\PY{l+s+s1}{fro}\PY{l+s+s1}{\PYZsq{}}\PY{p}{)}
\PY{n+nb}{print}\PY{p}{(}\PY{l+s+s1}{\PYZsq{}}\PY{l+s+s1}{No loop difference was: }\PY{l+s+si}{\PYZpc{}f}\PY{l+s+s1}{\PYZsq{}} \PY{o}{\PYZpc{}} \PY{p}{(}\PY{n}{difference}\PY{p}{,} \PY{p}{)}\PY{p}{)}
\PY{k}{if} \PY{n}{difference} \PY{o}{\PYZlt{}} \PY{l+m+mf}{0.001}\PY{p}{:}
    \PY{n+nb}{print}\PY{p}{(}\PY{l+s+s1}{\PYZsq{}}\PY{l+s+s1}{Good! The distance matrices are the same}\PY{l+s+s1}{\PYZsq{}}\PY{p}{)}
\PY{k}{else}\PY{p}{:}
    \PY{n+nb}{print}\PY{p}{(}\PY{l+s+s1}{\PYZsq{}}\PY{l+s+s1}{Uh\PYZhy{}oh! The distance matrices are different}\PY{l+s+s1}{\PYZsq{}}\PY{p}{)}
\end{Verbatim}
\end{tcolorbox}

    \begin{Verbatim}[commandchars=\\\{\}]
No loop difference was: 0.000000
Good! The distance matrices are the same
    \end{Verbatim}

    \begin{tcolorbox}[breakable, size=fbox, boxrule=1pt, pad at break*=1mm,colback=cellbackground, colframe=cellborder]
\prompt{In}{incolor}{ }{\boxspacing}
\begin{Verbatim}[commandchars=\\\{\}]
\PY{c+c1}{\PYZsh{} 让我们比较一下三种实现方式的速度}
\PY{k}{def} \PY{n+nf}{time\PYZus{}function}\PY{p}{(}\PY{n}{f}\PY{p}{,} \PY{o}{*}\PY{n}{args}\PY{p}{)}\PY{p}{:}
    \PY{l+s+sd}{\PYZdq{}\PYZdq{}\PYZdq{}}
\PY{l+s+sd}{    Call a function f with args and return the time (in seconds) that it took to execute.}
\PY{l+s+sd}{    \PYZdq{}\PYZdq{}\PYZdq{}}
    \PY{k+kn}{import} \PY{n+nn}{time}
    \PY{n}{tic} \PY{o}{=} \PY{n}{time}\PY{o}{.}\PY{n}{time}\PY{p}{(}\PY{p}{)}
    \PY{n}{f}\PY{p}{(}\PY{o}{*}\PY{n}{args}\PY{p}{)}
    \PY{n}{toc} \PY{o}{=} \PY{n}{time}\PY{o}{.}\PY{n}{time}\PY{p}{(}\PY{p}{)}
    \PY{k}{return} \PY{n}{toc} \PY{o}{\PYZhy{}} \PY{n}{tic}

\PY{n}{two\PYZus{}loop\PYZus{}time} \PY{o}{=} \PY{n}{time\PYZus{}function}\PY{p}{(}\PY{n}{classifier}\PY{o}{.}\PY{n}{compute\PYZus{}distances\PYZus{}two\PYZus{}loops}\PY{p}{,} \PY{n}{X\PYZus{}test}\PY{p}{)}
\PY{n+nb}{print}\PY{p}{(}\PY{l+s+s1}{\PYZsq{}}\PY{l+s+s1}{Two loop version took }\PY{l+s+si}{\PYZpc{}f}\PY{l+s+s1}{ seconds}\PY{l+s+s1}{\PYZsq{}} \PY{o}{\PYZpc{}} \PY{n}{two\PYZus{}loop\PYZus{}time}\PY{p}{)}

\PY{n}{one\PYZus{}loop\PYZus{}time} \PY{o}{=} \PY{n}{time\PYZus{}function}\PY{p}{(}\PY{n}{classifier}\PY{o}{.}\PY{n}{compute\PYZus{}distances\PYZus{}one\PYZus{}loop}\PY{p}{,} \PY{n}{X\PYZus{}test}\PY{p}{)}
\PY{n+nb}{print}\PY{p}{(}\PY{l+s+s1}{\PYZsq{}}\PY{l+s+s1}{One loop version took }\PY{l+s+si}{\PYZpc{}f}\PY{l+s+s1}{ seconds}\PY{l+s+s1}{\PYZsq{}} \PY{o}{\PYZpc{}} \PY{n}{one\PYZus{}loop\PYZus{}time}\PY{p}{)}

\PY{n}{no\PYZus{}loop\PYZus{}time} \PY{o}{=} \PY{n}{time\PYZus{}function}\PY{p}{(}\PY{n}{classifier}\PY{o}{.}\PY{n}{compute\PYZus{}distances\PYZus{}no\PYZus{}loops}\PY{p}{,} \PY{n}{X\PYZus{}test}\PY{p}{)}
\PY{n+nb}{print}\PY{p}{(}\PY{l+s+s1}{\PYZsq{}}\PY{l+s+s1}{No loop version took }\PY{l+s+si}{\PYZpc{}f}\PY{l+s+s1}{ seconds}\PY{l+s+s1}{\PYZsq{}} \PY{o}{\PYZpc{}} \PY{n}{no\PYZus{}loop\PYZus{}time}\PY{p}{)}

\PY{c+c1}{\PYZsh{} 你应该会看到使用完全矢量化的实现会有明显更佳的性能!}

\PY{c+c1}{\PYZsh{} 注意:在部分计算机上,当您从两层循环转到单层循环时,}
\PY{c+c1}{\PYZsh{} 您可能看不到速度的提升,甚至可能会看到速度变慢。}
\end{Verbatim}
\end{tcolorbox}

    \begin{Verbatim}[commandchars=\\\{\}]
Two loop version took 40.961476 seconds
One loop version took 39.660156 seconds
No loop version took 0.575312 seconds
    \end{Verbatim}

    \hypertarget{ux4ea4ux53c9ux9a8cux8bc1}{%
\subsubsection{交叉验证}\label{ux4ea4ux53c9ux9a8cux8bc1}}

我们已经实现了kNN分类器,并且可以设置k =
5。现在,将通过交叉验证来确定此超参数的最佳值。

    \begin{tcolorbox}[breakable, size=fbox, boxrule=1pt, pad at break*=1mm,colback=cellbackground, colframe=cellborder]
\prompt{In}{incolor}{ }{\boxspacing}
\begin{Verbatim}[commandchars=\\\{\}]
\PY{n}{num\PYZus{}folds} \PY{o}{=} \PY{l+m+mi}{5}
\PY{n}{k\PYZus{}choices} \PY{o}{=} \PY{p}{[}\PY{l+m+mi}{1}\PY{p}{,} \PY{l+m+mi}{3}\PY{p}{,} \PY{l+m+mi}{5}\PY{p}{,} \PY{l+m+mi}{8}\PY{p}{,} \PY{l+m+mi}{10}\PY{p}{,} \PY{l+m+mi}{12}\PY{p}{,} \PY{l+m+mi}{15}\PY{p}{,} \PY{l+m+mi}{20}\PY{p}{,} \PY{l+m+mi}{50}\PY{p}{,} \PY{l+m+mi}{100}\PY{p}{]}

\PY{n}{X\PYZus{}train\PYZus{}folds} \PY{o}{=} \PY{p}{[}\PY{p}{]}
\PY{n}{y\PYZus{}train\PYZus{}folds} \PY{o}{=} \PY{p}{[}\PY{p}{]}
\PY{c+c1}{\PYZsh{}\PYZsh{}\PYZsh{}\PYZsh{}\PYZsh{}\PYZsh{}\PYZsh{}\PYZsh{}\PYZsh{}\PYZsh{}\PYZsh{}\PYZsh{}\PYZsh{}\PYZsh{}\PYZsh{}\PYZsh{}\PYZsh{}\PYZsh{}\PYZsh{}\PYZsh{}\PYZsh{}\PYZsh{}\PYZsh{}\PYZsh{}\PYZsh{}\PYZsh{}\PYZsh{}\PYZsh{}\PYZsh{}\PYZsh{}\PYZsh{}\PYZsh{}\PYZsh{}\PYZsh{}\PYZsh{}\PYZsh{}\PYZsh{}\PYZsh{}\PYZsh{}\PYZsh{}\PYZsh{}\PYZsh{}\PYZsh{}\PYZsh{}\PYZsh{}\PYZsh{}\PYZsh{}\PYZsh{}\PYZsh{}\PYZsh{}\PYZsh{}\PYZsh{}\PYZsh{}\PYZsh{}\PYZsh{}\PYZsh{}\PYZsh{}\PYZsh{}\PYZsh{}\PYZsh{}\PYZsh{}\PYZsh{}\PYZsh{}\PYZsh{}\PYZsh{}\PYZsh{}\PYZsh{}\PYZsh{}\PYZsh{}\PYZsh{}\PYZsh{}\PYZsh{}\PYZsh{}\PYZsh{}\PYZsh{}\PYZsh{}\PYZsh{}\PYZsh{}\PYZsh{}\PYZsh{}}
\PY{c+c1}{\PYZsh{} 需要完成的事情: }
\PY{c+c1}{\PYZsh{} 将训练数据分成多个部分。拆分后,X\PYZus{}train\PYZus{}folds和y\PYZus{}train\PYZus{}folds均应为长度为num\PYZus{}folds的列表,}
\PY{c+c1}{\PYZsh{} 其中y\PYZus{}train\PYZus{}folds [i]是X\PYZus{}train\PYZus{}folds [i]中各点的标签向量。}
\PY{c+c1}{\PYZsh{} 提示:查阅numpy的array\PYZus{}split函数。                             }
\PY{c+c1}{\PYZsh{}\PYZsh{}\PYZsh{}\PYZsh{}\PYZsh{}\PYZsh{}\PYZsh{}\PYZsh{}\PYZsh{}\PYZsh{}\PYZsh{}\PYZsh{}\PYZsh{}\PYZsh{}\PYZsh{}\PYZsh{}\PYZsh{}\PYZsh{}\PYZsh{}\PYZsh{}\PYZsh{}\PYZsh{}\PYZsh{}\PYZsh{}\PYZsh{}\PYZsh{}\PYZsh{}\PYZsh{}\PYZsh{}\PYZsh{}\PYZsh{}\PYZsh{}\PYZsh{}\PYZsh{}\PYZsh{}\PYZsh{}\PYZsh{}\PYZsh{}\PYZsh{}\PYZsh{}\PYZsh{}\PYZsh{}\PYZsh{}\PYZsh{}\PYZsh{}\PYZsh{}\PYZsh{}\PYZsh{}\PYZsh{}\PYZsh{}\PYZsh{}\PYZsh{}\PYZsh{}\PYZsh{}\PYZsh{}\PYZsh{}\PYZsh{}\PYZsh{}\PYZsh{}\PYZsh{}\PYZsh{}\PYZsh{}\PYZsh{}\PYZsh{}\PYZsh{}\PYZsh{}\PYZsh{}\PYZsh{}\PYZsh{}\PYZsh{}\PYZsh{}\PYZsh{}\PYZsh{}\PYZsh{}\PYZsh{}\PYZsh{}\PYZsh{}\PYZsh{}\PYZsh{}\PYZsh{}}
\PY{c+c1}{\PYZsh{} *****START OF YOUR CODE (DO NOT DELETE/MODIFY THIS LINE)*****}

\PY{n}{y\PYZus{}train\PYZus{}} \PY{o}{=} \PY{n}{y\PYZus{}train}\PY{o}{.}\PY{n}{reshape}\PY{p}{(}\PY{o}{\PYZhy{}}\PY{l+m+mi}{1}\PY{p}{,} \PY{l+m+mi}{1}\PY{p}{)}
\PY{n}{X\PYZus{}train\PYZus{}folds} \PY{p}{,} \PY{n}{y\PYZus{}train\PYZus{}folds} \PY{o}{=} \PY{n}{np}\PY{o}{.}\PY{n}{array\PYZus{}split}\PY{p}{(}\PY{n}{X\PYZus{}train}\PY{p}{,} \PY{l+m+mi}{5}\PY{p}{)}\PY{p}{,} \PY{n}{np}\PY{o}{.}\PY{n}{array\PYZus{}split}\PY{p}{(}\PY{n}{y\PYZus{}train\PYZus{}}\PY{p}{,} \PY{l+m+mi}{5}\PY{p}{)}

\PY{c+c1}{\PYZsh{} *****END OF YOUR CODE (DO NOT DELETE/MODIFY THIS LINE)*****}

\PY{c+c1}{\PYZsh{} A dictionary holding the accuracies for different values of k that we find when running cross\PYZhy{}validation.}
\PY{c+c1}{\PYZsh{} 一个字典,存储我们进行交叉验证时不同k的值的精度。}
\PY{c+c1}{\PYZsh{} 运行交叉验证后,k\PYZus{}to\PYZus{}accuracies[k]应该是长度为num\PYZus{}folds的列表,存储了k值下的精度值。}
\PY{n}{k\PYZus{}to\PYZus{}accuracies} \PY{o}{=} \PY{p}{\PYZob{}}\PY{p}{\PYZcb{}}


\PY{c+c1}{\PYZsh{}\PYZsh{}\PYZsh{}\PYZsh{}\PYZsh{}\PYZsh{}\PYZsh{}\PYZsh{}\PYZsh{}\PYZsh{}\PYZsh{}\PYZsh{}\PYZsh{}\PYZsh{}\PYZsh{}\PYZsh{}\PYZsh{}\PYZsh{}\PYZsh{}\PYZsh{}\PYZsh{}\PYZsh{}\PYZsh{}\PYZsh{}\PYZsh{}\PYZsh{}\PYZsh{}\PYZsh{}\PYZsh{}\PYZsh{}\PYZsh{}\PYZsh{}\PYZsh{}\PYZsh{}\PYZsh{}\PYZsh{}\PYZsh{}\PYZsh{}\PYZsh{}\PYZsh{}\PYZsh{}\PYZsh{}\PYZsh{}\PYZsh{}\PYZsh{}\PYZsh{}\PYZsh{}\PYZsh{}\PYZsh{}\PYZsh{}\PYZsh{}\PYZsh{}\PYZsh{}\PYZsh{}\PYZsh{}\PYZsh{}\PYZsh{}\PYZsh{}\PYZsh{}\PYZsh{}\PYZsh{}\PYZsh{}\PYZsh{}\PYZsh{}\PYZsh{}\PYZsh{}\PYZsh{}\PYZsh{}\PYZsh{}\PYZsh{}\PYZsh{}\PYZsh{}\PYZsh{}\PYZsh{}\PYZsh{}\PYZsh{}\PYZsh{}\PYZsh{}\PYZsh{}\PYZsh{}}
\PY{c+c1}{\PYZsh{} 需要完成的事情:  }
\PY{c+c1}{\PYZsh{} 执行k的交叉验证,以找到k的最佳值。}
\PY{c+c1}{\PYZsh{} 对于每个可能的k值,运行k\PYZhy{}最近邻算法 num\PYZus{}folds 次,}
\PY{c+c1}{\PYZsh{} 在每次循环下,你都会用所有拆分的数据(除了其中一个需要作为验证集)作为训练数据。}
\PY{c+c1}{\PYZsh{} 然后存储所有的精度结果到k\PYZus{}to\PYZus{}accuracies[k]中。                              }
\PY{c+c1}{\PYZsh{}\PYZsh{}\PYZsh{}\PYZsh{}\PYZsh{}\PYZsh{}\PYZsh{}\PYZsh{}\PYZsh{}\PYZsh{}\PYZsh{}\PYZsh{}\PYZsh{}\PYZsh{}\PYZsh{}\PYZsh{}\PYZsh{}\PYZsh{}\PYZsh{}\PYZsh{}\PYZsh{}\PYZsh{}\PYZsh{}\PYZsh{}\PYZsh{}\PYZsh{}\PYZsh{}\PYZsh{}\PYZsh{}\PYZsh{}\PYZsh{}\PYZsh{}\PYZsh{}\PYZsh{}\PYZsh{}\PYZsh{}\PYZsh{}\PYZsh{}\PYZsh{}\PYZsh{}\PYZsh{}\PYZsh{}\PYZsh{}\PYZsh{}\PYZsh{}\PYZsh{}\PYZsh{}\PYZsh{}\PYZsh{}\PYZsh{}\PYZsh{}\PYZsh{}\PYZsh{}\PYZsh{}\PYZsh{}\PYZsh{}\PYZsh{}\PYZsh{}\PYZsh{}\PYZsh{}\PYZsh{}\PYZsh{}\PYZsh{}\PYZsh{}\PYZsh{}\PYZsh{}\PYZsh{}\PYZsh{}\PYZsh{}\PYZsh{}\PYZsh{}\PYZsh{}\PYZsh{}\PYZsh{}\PYZsh{}\PYZsh{}\PYZsh{}\PYZsh{}\PYZsh{}\PYZsh{}}
\PY{c+c1}{\PYZsh{} *****START OF YOUR CODE (DO NOT DELETE/MODIFY THIS LINE)*****}

\PY{c+c1}{\PYZsh{} 交叉验证。有时候,训练集数量较小(因此验证集的数量更小),人们会使用一种被称为}
\PY{c+c1}{\PYZsh{} 交叉验证的方法,这种方法更加复杂些。还是用刚才的例子,如果是交叉验证集,我们就}
\PY{c+c1}{\PYZsh{} 不是取1000个图像,而是将训练集平均分成5份,其中4份用来训练,1份用来验证。然后}
\PY{c+c1}{\PYZsh{} 我们循环着取其中4份来训练,其中1份来验证,最后取所有5次验证结果的平均值作为算}
\PY{c+c1}{\PYZsh{} 法验证结果。}

\PY{k}{for} \PY{n}{k\PYZus{}} \PY{o+ow}{in} \PY{n}{k\PYZus{}choices}\PY{p}{:}
    \PY{n}{k\PYZus{}to\PYZus{}accuracies}\PY{o}{.}\PY{n}{setdefault}\PY{p}{(}\PY{n}{k\PYZus{}}\PY{p}{,} \PY{p}{[}\PY{p}{]}\PY{p}{)}
\PY{k}{for} \PY{n}{i} \PY{o+ow}{in} \PY{n+nb}{range}\PY{p}{(}\PY{n}{num\PYZus{}folds}\PY{p}{)}\PY{p}{:}
    \PY{n}{classifier} \PY{o}{=} \PY{n}{KNearestNeighbor}\PY{p}{(}\PY{p}{)}
    \PY{n}{X\PYZus{}val\PYZus{}train} \PY{o}{=} \PY{n}{np}\PY{o}{.}\PY{n}{vstack}\PY{p}{(}\PY{n}{X\PYZus{}train\PYZus{}folds}\PY{p}{[}\PY{l+m+mi}{0}\PY{p}{:}\PY{n}{i}\PY{p}{]} \PY{o}{+} \PY{n}{X\PYZus{}train\PYZus{}folds}\PY{p}{[}\PY{n}{i}\PY{o}{+}\PY{l+m+mi}{1}\PY{p}{:}\PY{p}{]}\PY{p}{)}
    \PY{n}{y\PYZus{}val\PYZus{}train} \PY{o}{=} \PY{n}{np}\PY{o}{.}\PY{n}{vstack}\PY{p}{(}\PY{n}{y\PYZus{}train\PYZus{}folds}\PY{p}{[}\PY{l+m+mi}{0}\PY{p}{:}\PY{n}{i}\PY{p}{]} \PY{o}{+} \PY{n}{y\PYZus{}train\PYZus{}folds}\PY{p}{[}\PY{n}{i}\PY{o}{+}\PY{l+m+mi}{1}\PY{p}{:}\PY{p}{]}\PY{p}{)}
    \PY{n}{y\PYZus{}val\PYZus{}train} \PY{o}{=} \PY{n}{y\PYZus{}val\PYZus{}train}\PY{p}{[}\PY{p}{:}\PY{p}{,}\PY{l+m+mi}{0}\PY{p}{]}
    \PY{n}{classifier}\PY{o}{.}\PY{n}{train}\PY{p}{(}\PY{n}{X\PYZus{}val\PYZus{}train}\PY{p}{,} \PY{n}{y\PYZus{}val\PYZus{}train}\PY{p}{)}
    \PY{k}{for} \PY{n}{k\PYZus{}} \PY{o+ow}{in} \PY{n}{k\PYZus{}choices}\PY{p}{:}
        \PY{n}{y\PYZus{}val\PYZus{}pred} \PY{o}{=} \PY{n}{classifier}\PY{o}{.}\PY{n}{predict}\PY{p}{(}\PY{n}{X\PYZus{}train\PYZus{}folds}\PY{p}{[}\PY{n}{i}\PY{p}{]}\PY{p}{,} \PY{n}{k}\PY{o}{=}\PY{n}{k\PYZus{}}\PY{p}{)}
        \PY{n}{num\PYZus{}correct} \PY{o}{=} \PY{n}{np}\PY{o}{.}\PY{n}{sum}\PY{p}{(}\PY{n}{y\PYZus{}val\PYZus{}pred} \PY{o}{==} \PY{n}{y\PYZus{}train\PYZus{}folds}\PY{p}{[}\PY{n}{i}\PY{p}{]}\PY{p}{[}\PY{p}{:}\PY{p}{,}\PY{l+m+mi}{0}\PY{p}{]}\PY{p}{)}
        \PY{n}{accuracy} \PY{o}{=} \PY{n+nb}{float}\PY{p}{(}\PY{n}{num\PYZus{}correct}\PY{p}{)} \PY{o}{/} \PY{n+nb}{len}\PY{p}{(}\PY{n}{y\PYZus{}val\PYZus{}pred}\PY{p}{)}
        \PY{n}{k\PYZus{}to\PYZus{}accuracies}\PY{p}{[}\PY{n}{k\PYZus{}}\PY{p}{]} \PY{o}{=} \PY{n}{k\PYZus{}to\PYZus{}accuracies}\PY{p}{[}\PY{n}{k\PYZus{}}\PY{p}{]} \PY{o}{+} \PY{p}{[}\PY{n}{accuracy}\PY{p}{]}
\PY{c+c1}{\PYZsh{} *****END OF YOUR CODE (DO NOT DELETE/MODIFY THIS LINE)*****}

\PY{c+c1}{\PYZsh{} 打印出计算的精度}
\PY{k}{for} \PY{n}{k} \PY{o+ow}{in} \PY{n+nb}{sorted}\PY{p}{(}\PY{n}{k\PYZus{}to\PYZus{}accuracies}\PY{p}{)}\PY{p}{:}
    \PY{k}{for} \PY{n}{accuracy} \PY{o+ow}{in} \PY{n}{k\PYZus{}to\PYZus{}accuracies}\PY{p}{[}\PY{n}{k}\PY{p}{]}\PY{p}{:}
        \PY{n+nb}{print}\PY{p}{(}\PY{l+s+s1}{\PYZsq{}}\PY{l+s+s1}{k = }\PY{l+s+si}{\PYZpc{}d}\PY{l+s+s1}{, accuracy = }\PY{l+s+si}{\PYZpc{}f}\PY{l+s+s1}{\PYZsq{}} \PY{o}{\PYZpc{}} \PY{p}{(}\PY{n}{k}\PY{p}{,} \PY{n}{accuracy}\PY{p}{)}\PY{p}{)}
\end{Verbatim}
\end{tcolorbox}

    \begin{Verbatim}[commandchars=\\\{\}]
k = 1, accuracy = 0.263000
k = 1, accuracy = 0.257000
k = 1, accuracy = 0.264000
k = 1, accuracy = 0.278000
k = 1, accuracy = 0.266000
k = 3, accuracy = 0.239000
k = 3, accuracy = 0.249000
k = 3, accuracy = 0.240000
k = 3, accuracy = 0.266000
k = 3, accuracy = 0.254000
k = 5, accuracy = 0.248000
k = 5, accuracy = 0.266000
k = 5, accuracy = 0.280000
k = 5, accuracy = 0.292000
k = 5, accuracy = 0.280000
k = 8, accuracy = 0.262000
k = 8, accuracy = 0.282000
k = 8, accuracy = 0.273000
k = 8, accuracy = 0.290000
k = 8, accuracy = 0.273000
k = 10, accuracy = 0.265000
k = 10, accuracy = 0.296000
k = 10, accuracy = 0.276000
k = 10, accuracy = 0.284000
k = 10, accuracy = 0.280000
k = 12, accuracy = 0.260000
k = 12, accuracy = 0.295000
k = 12, accuracy = 0.279000
k = 12, accuracy = 0.283000
k = 12, accuracy = 0.280000
k = 15, accuracy = 0.252000
k = 15, accuracy = 0.289000
k = 15, accuracy = 0.278000
k = 15, accuracy = 0.282000
k = 15, accuracy = 0.274000
k = 20, accuracy = 0.270000
k = 20, accuracy = 0.279000
k = 20, accuracy = 0.279000
k = 20, accuracy = 0.282000
k = 20, accuracy = 0.285000
k = 50, accuracy = 0.271000
k = 50, accuracy = 0.288000
k = 50, accuracy = 0.278000
k = 50, accuracy = 0.269000
k = 50, accuracy = 0.266000
k = 100, accuracy = 0.256000
k = 100, accuracy = 0.270000
k = 100, accuracy = 0.263000
k = 100, accuracy = 0.256000
k = 100, accuracy = 0.263000
    \end{Verbatim}

    \begin{tcolorbox}[breakable, size=fbox, boxrule=1pt, pad at break*=1mm,colback=cellbackground, colframe=cellborder]
\prompt{In}{incolor}{ }{\boxspacing}
\begin{Verbatim}[commandchars=\\\{\}]
\PY{c+c1}{\PYZsh{} 绘制原始观察结果}
\PY{k}{for} \PY{n}{k} \PY{o+ow}{in} \PY{n}{k\PYZus{}choices}\PY{p}{:}
    \PY{n}{accuracies} \PY{o}{=} \PY{n}{k\PYZus{}to\PYZus{}accuracies}\PY{p}{[}\PY{n}{k}\PY{p}{]}
    \PY{n}{plt}\PY{o}{.}\PY{n}{scatter}\PY{p}{(}\PY{p}{[}\PY{n}{k}\PY{p}{]} \PY{o}{*} \PY{n+nb}{len}\PY{p}{(}\PY{n}{accuracies}\PY{p}{)}\PY{p}{,} \PY{n}{accuracies}\PY{p}{)}

\PY{c+c1}{\PYZsh{} 用与标准偏差相对应的误差线绘制趋势线}
\PY{n}{accuracies\PYZus{}mean} \PY{o}{=} \PY{n}{np}\PY{o}{.}\PY{n}{array}\PY{p}{(}\PY{p}{[}\PY{n}{np}\PY{o}{.}\PY{n}{mean}\PY{p}{(}\PY{n}{v}\PY{p}{)} \PY{k}{for} \PY{n}{k}\PY{p}{,}\PY{n}{v} \PY{o+ow}{in} \PY{n+nb}{sorted}\PY{p}{(}\PY{n}{k\PYZus{}to\PYZus{}accuracies}\PY{o}{.}\PY{n}{items}\PY{p}{(}\PY{p}{)}\PY{p}{)}\PY{p}{]}\PY{p}{)}
\PY{n}{accuracies\PYZus{}std} \PY{o}{=} \PY{n}{np}\PY{o}{.}\PY{n}{array}\PY{p}{(}\PY{p}{[}\PY{n}{np}\PY{o}{.}\PY{n}{std}\PY{p}{(}\PY{n}{v}\PY{p}{)} \PY{k}{for} \PY{n}{k}\PY{p}{,}\PY{n}{v} \PY{o+ow}{in} \PY{n+nb}{sorted}\PY{p}{(}\PY{n}{k\PYZus{}to\PYZus{}accuracies}\PY{o}{.}\PY{n}{items}\PY{p}{(}\PY{p}{)}\PY{p}{)}\PY{p}{]}\PY{p}{)}
\PY{n}{plt}\PY{o}{.}\PY{n}{errorbar}\PY{p}{(}\PY{n}{k\PYZus{}choices}\PY{p}{,} \PY{n}{accuracies\PYZus{}mean}\PY{p}{,} \PY{n}{yerr}\PY{o}{=}\PY{n}{accuracies\PYZus{}std}\PY{p}{)}
\PY{n}{plt}\PY{o}{.}\PY{n}{title}\PY{p}{(}\PY{l+s+s1}{\PYZsq{}}\PY{l+s+s1}{Cross\PYZhy{}validation on k}\PY{l+s+s1}{\PYZsq{}}\PY{p}{)}
\PY{n}{plt}\PY{o}{.}\PY{n}{xlabel}\PY{p}{(}\PY{l+s+s1}{\PYZsq{}}\PY{l+s+s1}{k}\PY{l+s+s1}{\PYZsq{}}\PY{p}{)}
\PY{n}{plt}\PY{o}{.}\PY{n}{ylabel}\PY{p}{(}\PY{l+s+s1}{\PYZsq{}}\PY{l+s+s1}{Cross\PYZhy{}validation accuracy}\PY{l+s+s1}{\PYZsq{}}\PY{p}{)}
\PY{n}{plt}\PY{o}{.}\PY{n}{show}\PY{p}{(}\PY{p}{)}
\end{Verbatim}
\end{tcolorbox}

    \begin{center}
    \adjustimage{max size={0.9\linewidth}{0.9\paperheight}}{output_27_0.png}
    \end{center}
    { \hspace*{\fill} \\}
    
    \begin{tcolorbox}[breakable, size=fbox, boxrule=1pt, pad at break*=1mm,colback=cellbackground, colframe=cellborder]
\prompt{In}{incolor}{ }{\boxspacing}
\begin{Verbatim}[commandchars=\\\{\}]
\PY{c+c1}{\PYZsh{} 根据上述交叉验证结果,为k选择最佳值,使用所有训练数据重新训练分类器,}
\PY{c+c1}{\PYZsh{} 并在测试中对其进行测试数据。您应该能够在测试数据上获得28%以上的准确性。}

\PY{n}{best\PYZus{}k} \PY{o}{=} \PY{n}{k\PYZus{}choices}\PY{p}{[}\PY{n}{accuracies\PYZus{}mean}\PY{o}{.}\PY{n}{argmax}\PY{p}{(}\PY{p}{)}\PY{p}{]}

\PY{n}{classifier} \PY{o}{=} \PY{n}{KNearestNeighbor}\PY{p}{(}\PY{p}{)}
\PY{n}{classifier}\PY{o}{.}\PY{n}{train}\PY{p}{(}\PY{n}{X\PYZus{}train}\PY{p}{,} \PY{n}{y\PYZus{}train}\PY{p}{)}
\PY{n}{y\PYZus{}test\PYZus{}pred} \PY{o}{=} \PY{n}{classifier}\PY{o}{.}\PY{n}{predict}\PY{p}{(}\PY{n}{X\PYZus{}test}\PY{p}{,} \PY{n}{k}\PY{o}{=}\PY{n}{best\PYZus{}k}\PY{p}{)}

\PY{c+c1}{\PYZsh{} Compute and display the accuracy}
\PY{n}{num\PYZus{}correct} \PY{o}{=} \PY{n}{np}\PY{o}{.}\PY{n}{sum}\PY{p}{(}\PY{n}{y\PYZus{}test\PYZus{}pred} \PY{o}{==} \PY{n}{y\PYZus{}test}\PY{p}{)}
\PY{n}{accuracy} \PY{o}{=} \PY{n+nb}{float}\PY{p}{(}\PY{n}{num\PYZus{}correct}\PY{p}{)} \PY{o}{/} \PY{n}{num\PYZus{}test}
\PY{n+nb}{print}\PY{p}{(}\PY{l+s+s1}{\PYZsq{}}\PY{l+s+s1}{Got }\PY{l+s+si}{\PYZpc{}d}\PY{l+s+s1}{ / }\PY{l+s+si}{\PYZpc{}d}\PY{l+s+s1}{ correct =\PYZgt{} accuracy: }\PY{l+s+si}{\PYZpc{}f}\PY{l+s+s1}{\PYZsq{}} \PY{o}{\PYZpc{}} \PY{p}{(}\PY{n}{num\PYZus{}correct}\PY{p}{,} \PY{n}{num\PYZus{}test}\PY{p}{,} \PY{n}{accuracy}\PY{p}{)}\PY{p}{)}
\end{Verbatim}
\end{tcolorbox}

    \begin{Verbatim}[commandchars=\\\{\}]
Got 141 / 500 correct => accuracy: 0.282000
    \end{Verbatim}

    \textbf{问题 3}

下列关于\(k\)-NN的陈述中哪些是在分类器中正确的设置,并且对所有的\(k\)都有效?选择所有符合条件的选项。

\begin{enumerate}
\def\labelenumi{\arabic{enumi}.}
\tightlist
\item
  k-NN分类器的决策边界是线性的。
\item
  1-NN的训练误差将始终低于5-NN。
\item
  1-NN的测试误差将始终低于5-NN。
\item
  使用k-NN分类器对测试示例进行分类所需的时间随训练集的大小而增加。
\item
  以上都不是。
\end{enumerate}

\(\color{blue}{\textit 你的回答:}\) 24

\(\color{blue}{\textit 你的解释:}\) 1NN训练不始终百分百吗
4是的,但是增加复杂度没这么高,因为可以noloops去写

    \begin{center}\rule{0.5\linewidth}{0.5pt}\end{center}

\hypertarget{ux91cdux8981}{%
\section{重要}\label{ux91cdux8981}}

这里是作业的结尾处,请执行以下步骤:

\begin{enumerate}
\def\labelenumi{\arabic{enumi}.}
\tightlist
\item
  点击\texttt{File\ -\textgreater{}\ Save}或者用\texttt{control+s}组合键,确保你最新的的notebook的作业已经保存到谷歌云。
\item
  执行以下代码确保 \texttt{.py} 文件保存回你的谷歌云。
\end{enumerate}

    \begin{tcolorbox}[breakable, size=fbox, boxrule=1pt, pad at break*=1mm,colback=cellbackground, colframe=cellborder]
\prompt{In}{incolor}{ }{\boxspacing}
\begin{Verbatim}[commandchars=\\\{\}]
\PY{k+kn}{import} \PY{n+nn}{os}

\PY{n}{FOLDER\PYZus{}TO\PYZus{}SAVE} \PY{o}{=} \PY{n}{os}\PY{o}{.}\PY{n}{path}\PY{o}{.}\PY{n}{join}\PY{p}{(}\PY{l+s+s1}{\PYZsq{}}\PY{l+s+s1}{drive/My Drive/}\PY{l+s+s1}{\PYZsq{}}\PY{p}{,} \PY{n}{FOLDERNAME}\PY{p}{)}
\PY{n}{FILES\PYZus{}TO\PYZus{}SAVE} \PY{o}{=} \PY{p}{[}\PY{l+s+s1}{\PYZsq{}}\PY{l+s+s1}{daseCV/classifiers/k\PYZus{}nearest\PYZus{}neighbor.py}\PY{l+s+s1}{\PYZsq{}}\PY{p}{]}

\PY{k}{for} \PY{n}{files} \PY{o+ow}{in} \PY{n}{FILES\PYZus{}TO\PYZus{}SAVE}\PY{p}{:}
  \PY{k}{with} \PY{n+nb}{open}\PY{p}{(}\PY{n}{os}\PY{o}{.}\PY{n}{path}\PY{o}{.}\PY{n}{join}\PY{p}{(}\PY{n}{FOLDER\PYZus{}TO\PYZus{}SAVE}\PY{p}{,} \PY{l+s+s1}{\PYZsq{}}\PY{l+s+s1}{/}\PY{l+s+s1}{\PYZsq{}}\PY{o}{.}\PY{n}{join}\PY{p}{(}\PY{n}{files}\PY{o}{.}\PY{n}{split}\PY{p}{(}\PY{l+s+s1}{\PYZsq{}}\PY{l+s+s1}{/}\PY{l+s+s1}{\PYZsq{}}\PY{p}{)}\PY{p}{[}\PY{l+m+mi}{1}\PY{p}{:}\PY{p}{]}\PY{p}{)}\PY{p}{)}\PY{p}{,} \PY{l+s+s1}{\PYZsq{}}\PY{l+s+s1}{w}\PY{l+s+s1}{\PYZsq{}}\PY{p}{)} \PY{k}{as} \PY{n}{f}\PY{p}{:}
    \PY{n}{f}\PY{o}{.}\PY{n}{write}\PY{p}{(}\PY{l+s+s1}{\PYZsq{}}\PY{l+s+s1}{\PYZsq{}}\PY{o}{.}\PY{n}{join}\PY{p}{(}\PY{n+nb}{open}\PY{p}{(}\PY{n}{files}\PY{p}{)}\PY{o}{.}\PY{n}{readlines}\PY{p}{(}\PY{p}{)}\PY{p}{)}\PY{p}{)}
\end{Verbatim}
\end{tcolorbox}


    % Add a bibliography block to the postdoc
    
    
    
\end{document}
